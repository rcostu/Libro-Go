
\chapter{Concurrencia y comunicación}

A medida que la informática avanza, se va haciendo más necesario el uso de
elementos de comunicación y concurrencia. La comunicación es algo necesario para
no formar parte de un sistema aislado, y así poder comunicar nuestro programa
con otros programas, que puede que estén o no en nuestra misma máquina.\\

La concurrencia, por otra parte, nos permite evitar situaciones indeseadas
y realizar programación paralela, haciendo que el aprovechamiento de los
recursos sea mucho más eficaz. Go es un lenguaje creado con dos fines: uno de
ellos es la simplicidad, y el otro es su eminente enfoque para sistemas
concurrentes.

\section{Goroutines}

	\subsection{Definición}
	
	Podríamos definir de manera formal una Goroutine como: \\

	\textit{''Una función de Go que permite ejecutar otra función de manera
	paralela al programa y en el mismo espacio de direcciones que realiza la
	llamada a la Goroutine.''}\\

	Todos los programas en Go constan de una o varias Goroutines.\\

	Aunque pueda llevar a confusión, una Goroutine no es lo mismo que un thread,
	un proceso, etc. Una Goroutine es una Goroutine.\\

	Evidentemente, y aunque pueda parecer un sistema perfecto, tiene sus
	problemas de concurrencia asociados que se irán viendo más adelante. Por
	ahora asumamos que funciona tal y como se ha dicho.  \clearpage
	\subsection{Cómo crear una Goroutine}
	
	Crear una Goroutine es quizá lo más sencillo que haya. Basta con invocar una
	función anteponiéndole la palabra reservada \textit{go}:
	
	\begin{verbatim} func IsReady (what strgin, minutes int64) { time.Sleep
	(minutes * 60 * 1e9) fmt.Println (what, "está listo") }
	   
		go IsReady ("té", 6) go IsReady ("café", 2) fmt.Println ("Estoy
		esperando...") \end{verbatim}
	
	El ejemplo anterior imprimiría:
	
	\begin{verbatim} Estoy esperando...   (de forma inmediata) café está listo
	(2 minutos después) té está listo        (6 minutos después) \end{verbatim}
	
	\subsection{Verdades de las Goroutines}
	
	Las Goroutines son baratas, es decir, son muy fáciles de programar y de
	invocar.\\
	
	Las Goroutines terminan su ejecución retornando de su función superior,
	o simplemente saliendo en algún punto intermedio del programa puesto a tal
	propósito. También pueden llamar a \textit{runtime.Goexit()}, aunque es
	raramente necesario.\\
	
	Las Goroutines pueden ejecutarse concurrentemente en diferentes procesadores
	con memoria compartida. Asimismo no hay por qué preocuparse del tamaño de la
	pila a la hora de invocarlas.
	
	\subsection{Pila de una Goroutine}
	
	En \textit{gccgo}, al menos por ahora, las goroutines se compilan como si
	fueran pthreads. Sus pilas son grandes. Se pretende cambiar esto en un
	futuro próximo pero no se ha realizado todavía ya que necesitan realizarse
	cambios en el propio \textit{gcc}.\\
	
	Usando el compilador \textit{6g}, las pilas son pequeñas (unos pocos kB)
	pero crecen a medida que sea necesario. De esta forma, las goroutines usan
	muy poca memoria pero si es necesario pueden de forma dinámica hacer crecer
	la pila para invocar a muchas más funciones.\\
	
	Esto se ha diseñado y programado así para que el programador no tenga que
	preocuparse sobre el límite de la pila y la tarea le resulte mucho más
	cómoda.
	
	\subsection{Scheduling}
	
	Las goroutines están multiplexadas entre múltiples threads del sistema.
	Cuando una goroutine ejecuta una llamada bloqueante al sistema, el resto de
	goroutines no se bloquean y continúan su ejecución.\\
	
	Se pretende realizar este mismo efecto para las goroutines dependientes de
	la CPU, pero por ahora si se quiere un paralelismo a nivel de usuario se
	necesita dar un valor a la variable de entorno \textit{\$GOMAXPROCS}, o se
	puede llamar en un programa a \textit{runtime.GOMAXPROCS(n)}.\\
	
	\textit{GOMAXPROCS} le dice al planificador en tiempo de ejecución cuántas
	goroutines que no sean bloqueantes por llamadas al sistema, pueden correr
	a la vez. Esto implica, que para hacer una prueba real de paralelismo,
	necesitaremos adecuar el valor de \textit{\$GOMAXPROCS} al número de
	goroutines que tenga nuestro programa.\\
	
	¿Cómo se puede dar un valor a dicha variable? Sencillo. En un terminal del
	sistema basta con poner las siguientes líneas:
	
	\begin{verbatim} $ GOMAXPROCS=4 $ export GOMAXPROCS \end{verbatim}

\section{Channels\label{channels}}

A menos que dos goroutines puedan comunicarse, éstas no pueden coordinarse
o sincronizarse de ninguna forma. Por ello Go tiene un tipo denominado
\textit{channel} que provee comunicación y sincronización. También posee de
estructuras de control especiales para los canales que hacen fácil la
programación concurrente.

	\subsection{El tipo channel}
	
	En su forma más simple el tipo se declara como:
	
	\begin{verbatim} chan tipo_elemento \end{verbatim}
	
	Dada una variable de ese estilo, puedes enviar y recibir elementos del tipo
	\textit{tipo\textunderscore elemento}.\\
	
	Los canales son un tipo de referencia, lo que implica que se pude asignar
	una variable \textit{chan} a otra y que ambas variables accedan al mismo
	canal. Esto también hace referencia a que hay que usar \textit{make} para
	alojar un canal:
	
	\begin{verbatim} var c = make (chan int) \end{verbatim} \clearpage
	\subsection{El operador $<-$}
	
	El operador $<-$ (más comúnmente conocido como ''flechita'') es un operador
	que nos permite pasar datos a los canales. La flecha apunta en la dirección
	en que se pretende que sea el flujo de datos.\\
	
	Como operador binario, $<-$ envía a un canal:
	
	\begin{verbatim} var c chan int c <- 1      // envía 1 al canal
	c \end{verbatim}
	
	Como un operador unario prefijo, $<-$ recibe datos de un canal:
	
	\begin{verbatim} v = <- c    // Recibe un valor de c y se lo asigna a v.  <-
	c        // Recibe un valor de c, descartándolo i := <- c   // Recibe un
	valor de c inicializando i \end{verbatim}
	
	\subsection{Semántica de canales}
	
	Por defecto la comunicación en Go se realiza de forma síncrona. Más tarde
	abordaremos la comunicacion asíncrona. Que la comunicación sea síncrona
	significa:
	
	\begin{enumerate} \item Una operación send en un canal bloquea la ejecución
	hasta que se ejecuta una operación receive en el mismo canal.  \item Una
	operación receive en un canal bloquea la ejecución hasta que una operación
	send es ejecutada en el mismo canal.  \end{enumerate}
	
	De esta forma, la comunicación es además una forma de sincronización:
	2 goroutines intercambiando datos a través de un canal, se sincronizan en el
	momento en que la comunicación es efectiva.
	
	\subsection{Ejemplo de comunicación}
	
	Veamos un ejemplo bastante tonto de comunicación pero que servirá como toma
	de contacto con la sintaxis de los canales:
	
	\begin{verbatim} func pump (ch chan int) { for i := 0; ; i++ { ch <- i } }
	   
		ch1 := make (chan int) go pump (ch1)       // pump se bloquea;
		ejecutamos: fmt.Println (<-ch1)     // Imprime 0
	   
	   
		func suck (ch chan int) { for { fmt.Println (<-ch) } }
	   
		go suck (ch1)    // Miles de números se imprimen
	   
		// Todavía se puede colar la función principal y pedir un valor
		fmt.Println(<-ch)    // Imprime un valor cualquiera (314159)
		\end{verbatim}
	
	\subsection{Funciones que devuelven canales}
	
	En el ejemplo anterior, \textit{pump} era una función generadora de valores
	para ser recogidos por otra función. El ejemplo anterior se puede
	simplificar bastante en el momento en el que se cree una función que pueda
	devolver el canal en sí para que se puedan recoger los datos.
	
	\begin{verbatim} func pump() chan int { ch := make (chan int) go func()
	{ for i := 0; ; i++ { ch <- i } }() return ch }
	   
		stream := pump() fmt.Println(<-stream)   // Imprime 0 \end{verbatim}
	
	\subsection{Rangos y canales}
	
	La claúsula \textit{range} en bucles \textit{for} acepta un canal como
	operando, en cuyo caso el bucle \textit{for} itera sobre los valores
	recibidos por el canal.\\
	
	Anteriormente reescribimos la función pump(), con lo que ahora reescribimos
	la función \textit{suck} para que lance la goroutine también:
	
	\begin{verbatim} func suck (ch chan int) { go func() { for v:= range ch
	{ fmt.Println(v) } }() }
	   
		suck (pump())   // Ahora esta llamada no se bloquea \end{verbatim}
	
	\subsection{Cerrando un canal}
	
	Cuando queremos cerrar un canal por alguna razón usamos una función del
	lenguaje que nos permite realizar dicha acción.
	
	\begin{verbatim} close (ch) \end{verbatim}
	
	De igual forma, existe otra función que nos permite comprobar si un canal
	está cerrado o no, usando la función \textit{closed()}:
	
	\begin{verbatim} if closed (ch) { fmt.Println ("hecho") } \end{verbatim}
	
	Una vez que un canal está cerrado y que todos los valores enviados han sido
	recibidos, todas las operaciones receive que se realicen a continuación
	recuperarán un valor ''cero'' del canal.\\
	
	En la práctica, raramente se necesitará \textit{close}.\\
	
	Hay ciertas sutilezas respecto al cierre de un canal. La llamada
	a \textit{closed()} sólo funciona correctamente cuando se invoca después de
	haber recibido un valor ''cero'' en el canal. El bucle correcto para
	recorrer los datos de un canal y comprobar si se ha cerrado es:
	
	\begin{verbatim} for { v := <-ch if closed (ch) { break } fmt.Println (v)
	} \end{verbatim}
	
	Pero de una forma mucho más idiomática y simple, podemos usar la cláusula
	\textit{range} que realiza esa misma acción por nosotros:
	
	\begin{verbatim} for v := range ch { fmt.Println (v) } \end{verbatim}
	
	\subsection{Iteradores}
	
	Ahora que tenemos todas las piezas necesarias podemos construir un iterador
	para un contenedor. Aquí está el código para un \textit{Vector}: \clearpage
	\begin{verbatim} // Iteramos sobre todos los elementos
	   
		func (p *Vector) iterate (c chan Element) { for i, v := range p.a {   //
		p.a es un slice c <- v } close (c)     // Señala que no hay más valores
		}
	   
		// Iterador del canal
	   
		func (p *Vector) Iter() chan Element { c := make (chan Element) go
		p.iterate (c) return c } \end{verbatim}
	
	Ahora que el \textit{Vector} tiene un iterador, podemos usarlo:
	
	\begin{verbatim} vec := new (vector.Vector) for i := 0 i < 100; i++
	{ vec.Push (i*i) }
	   
		i:= 0 for x := range vec.Iter() { fmt.Printf ("vec[\%d] is \%d\n", i,
		x.(int)) i++ } \end{verbatim}
	
	\subsection{Direccionalidad de un canal}
	
	En su forma más simple una variable de tipo canal es una variable sin
	memoria (no posee un buffer), síncrona y que puede ser utilizada para enviar
	y recibir.\\
	
	Una variable de tipo canal puede ser anotada para especificar que únicamente
	puede enviar o recibir:
	
	\begin{verbatim} var recv_only <-chan int var send_only chan <- int
	\end{verbatim}
	
	Todos los canales son creados de forma bidireccional, pero podemos
	asignarlos a variables de canales direccionales. Esto es útil por ejemplo en
	las funciones, para tener seguridad en los tipos de los parámetros pasados:
	
	\begin{verbatim}
	
	
		func sink (ch <- chan int) { for { <- ch } }
	   
		func source (ch chan<- int) { for { ch <- 1 } }
	   
		var c = make (chan) int)  //bidireccional go source (c) go sink (c)
		\end{verbatim}
	
	\subsection{Canales síncronos}
	
	Los canales síncronos no disponen de un buffer. Las operaciones send no
	terminan hasta que un receive es ejecutado. Veamos un ejemplo de un canal
	síncrono:
	
	\begin{verbatim} c := make (chan int) go func () { time.Sleep (60 * 1e9)
	x := <- c fmt.Println ("recibido", x) }()
	   
		fmt.Println ("enviando", 10) c <- 10 fmt.Println ("enviado", 10)
		\end{verbatim}
	
	Salida:
	
	\begin{verbatim} enviando 10  (ocurre inmediatamente) enviado 10   (60s
	después, estas dos lineas aparecen) recibido 10 \end{verbatim}

	\subsection{Canales asíncronos}
	
	Un canal asíncrono con buffer puede ser creado pasando a \textit{make} un
	argumento, que será el número de elementos que tendrá el buffer.
	
	\begin{verbatim} c := make (chan int, 50) go func () { time.Sleep (60 * 1e9)
	x := <-c fmt.Println ("recibido", x) }()
	   
	   
		fmt.Println ("enviando", 10) c <- 10 fmt.Println ("enviado", 10)
		\end{verbatim}
	
	\begin{verbatim} enviando 10  (ocurre inmediatamente) enviado 10   (ahora)
	recibido 10   (60s después) \end{verbatim}
	
	\textbf{Nota.-} El buffer no forma parte del tipo canal, sino que es parte
	de la variable de tipo canal que se crea.
	
	\subsection{Probando la comunicación}
	
	Si nos preguntamos si un receive puede en un momento determinado ejecutar
	sin quedarse bloqueado, existe un método que permite hacerlo de forma
	atómica, probando y ejecutando, utilizando para tal efecto el sistema ''coma
	ok''.
	
	\begin{verbatim} v, ok = <-c   // ok = true si v recibe un valor
	\end{verbatim}
	
	¿Y qué hay de realizar una operación send sin bloquear? Hace falta
	averiguarlo y ejecutar. Para ello usamos una expresión booleana:
	
	\begin{verbatim} ok := c <- v
	   
		o también:
	   
		if c <- v { fmt.Println ("valor enviado") } \end{verbatim}
	
	Normalmente, se quiere una mayor flexibilidad, que conseguiremos con la
	siguiente estructura que veremos en el uso de canales.

\section{Select}

	\subsection{Definición}
	
	\textit{Select} es una estructura de control en Go, análoga a un switch
	asociado a comunicación. En un \textit{select} cada case debe ser una
	comunicación: send o receive.
	
	\begin{verbatim} var c1, c2 chan int
	   
		select { case v := <- c1: fmt.Printf ("recibido \%d de c1\n", v) case
		v := <- c2: fmt.Printf ("recibido \%d de c2\n", v) } \end{verbatim}
	
	\textit{Select} ejecuta un case que pueda ejecutarse de forma aleatoria. Si
	ninguno de los case es posible de ejecutar, se bloquea hasta que uno lo sea.
	Una cláusula \textit{default} es siempre ejecutable.
	
	\subsection{Semántica de select}
	
	Hagamos un rápido resumen del select:
	
	\begin{itemize} \item Cada case debe ser una comunicación \item Todas las
	expresiones del canal son evaluadas \item Todas las expresiones para ser
	enviadas son evaluadas \item Si alguna comunicación puede producirse, se
	produce; En otro caso se ignora.  \begin{itemize} \item Si existe una
	cláusula \textit{default}, es ésta la que ejecuta \item Si no existe
	\textit{default}, el bloque select se bloquea hasta que una comunicación
	pueda realizarse; no hay reevaluación de los canales y/o posibles valores.
	\end{itemize} \item Si hay múltiples casos listos, uno es seleccionado
	aleatoriamente. El resto no ejecuta.  \end{itemize}
	
	\subsection{Ejemplo: Generador aleatorio de bits}
	
	Un ejemplo bastante ilustrativo es el siguiente:
	
	\begin{verbatim} c := make (chan int) go func() { for { fmt.Println (<-c)
	} }()
	   
		for { select { case c <- 0:   // No hay sentencias, no existe el
		fall-through case c <- 1: } } \end{verbatim}
	
	Esto imprime: 0 1 1 0 0 1 1 1 0 1 ...

\section{Multiplexación}

Los canales son valores de ''primer tipo'', es decir, pueden ser enviados
a través de otros canales. Esta propiedad hace que sea fácil de escribir un
servicio multiplexador dado que el cliente puede proporcionar, junto con la
petición, el canal al que debe responder.

\begin{verbatim} chanOfChans := make (chan chan int) \end{verbatim} \clearpage
O de forma más típica:

\begin{verbatim} type Reply struct { ... } type Request struct { arg1, arg2,
arg3 some_type replyc chan *Reply } \end{verbatim}

Vamos a ver un ejemplo cliente-servidor a lo largo de los siguientes puntos.

	\subsection{El servidor}
	
	\begin{verbatim} type request struct { a, b int replyc chan int }
	   
		type binOp func (a, b int) int
	   
		func run (op binOp, req *request) { req.replyc <- op(req.a, req.b) }
	   
		func server (op binOp, service chan *request) { for { req := <- service
		// Aquí se aceptan las peticiones go run(op, req) } } \end{verbatim}
	
	Para comenzar el servidor lo hacemos de la siguiente forma:
	
	\begin{verbatim} func startServer (op binOp) chan *request { req := make
	(chan *request) go server (op, req) return req }
	   
		var adderChan = startServer( func a, b int) int { return a + b })
		\end{verbatim} \clearpage \subsection{El cliente}
	
	\begin{verbatim} func (r *request) String() string { return fmt.Sprintf
	("\%d+\%d=\%d", r.a, r.b, <-r.replyc) } req1 := &request{ 7, 8, make (chan
	int) } req2 := &request{ 17, 18, make (chan int) }
	   
		adderChan <- req1 adderChan <- req2
	   
		fmt.Println (req2, req1) \end{verbatim}
	
	\subsection{La contrapartida}
	
	En el ejemplo anterior el servidor ejecuta en un bucle infinito, con lo que
	siempre está esperando peticiones. Para terminarlo de manera correcta, se le
	puede señalar con un canal. El siguiente servidor tiene la misma
	funcionalidad pero con un canal \textit{quit} para terminar el proceso.
	
	\begin{verbatim} func server (op binOp, service chan *request, quit chan
	bool) { for { select{ case req := <- service:  // Aquí se aceptan las
	peticiones go run(op, req) case <- quit: return } } } \end{verbatim}
	
	El resto del código es básicamente el mismo, sólo que con un canal más:
	
	\begin{verbatim} func startServer (op binOp) (service chan *request, quit
	chan bool) { req := make (chan *request) quit = make (chan bool) go server
	(op, service, quit) return service, quit }
	   
		var adderChan, quitChan = startServer( func a, b int) int { return
		a + b }) \end{verbatim}
	
	El cliente permanece inalterado ya que puede terminar el servidor cuando
	quiera con sólo ejecutar la sentencia:
	
	\begin{verbatim} quitChan <- true \end{verbatim}

\section{Problemas de concurrencia}

Como siempre que se programa de forma concurrente existen un montón de problemas
pero Go trata de hacerse cargo de ellos.\\

El envío y la recepción (operaciones send y receive) son atómicas. La sentencia
\textit{select} está programada y definida muy cuidadosamente para que no
contenga errores.\\

El problema viene principalmente dado porque las goroutines ejecutan en memoria
compartida, las redes de comunicacción pueden colapsarse, que no existen
debuggers multihilo...\\

El consejo por parte de los desarrolladores del lenguaje es no programar como si
estuviéramos en C o C++, o incluso Java.\\

Los canales dan sincronización y comunicación en una misma estructura, lo cual
los hace muy potentes, pero también hacen que sea fácil razonar si se están
usando de manera correcta o no.\\

La regla es:\\

\textit{``No comunicar programas con memoria compartida, sino compartir memoria
comunicándose''}\\

El modelo básico a la hora de programar utilizando la filosofía cliente-servidor
se resume en dos conceptos:

\begin{enumerate} \item Usar un canal para enviar datos a una goroutine dedicada
como servidor, ya que si una única goroutine en cada instante tiene un puntero
a los datos, no hay problemas de concurrencia.  \item Usar el modelo de
programación ``un hilo por cliente'' de forma generalizada, usado desde la
década de los 80 y que funciona bastante bien.  \end{enumerate}

\clearpage \thispagestyle{empty} \ \\ \newpage
