
\chapter{Estructuras de control}

A lo largo de este capítulo, vamos a ir viendo las distintas estructuras de
control que posee Go. Antes de comenzar, hay que realizar un apunte,
concerniente especialmente a las sentencias \textit{if}, \textit{for}
y \textit{switch}:

\begin{itemize} \item Son similares a las declaraciones homónimas en C, pero
poseen diferencias significativas.  \item Como ya se ha apuntado, no utilizan
paréntesis y las llaves son obligatorias.  \item Las tres sentencias admiten
instrucciones de inicialización (operador ':=').  \end{itemize}

\section{Sentencia condicional: if\label{if}}

La sentencia condicional \textit{if} permite ejecutar una rama de código u otra
dependiendo de la evaluación de la condición.\\

Esta sentencia, es similar a la existente en otros lenguajes, pero tiene el
problema de que el \textit{else} tiene que estar en la misma línea que la llave
de cierre del \textit{if}.\\

Además admite instrucciones de inicialización, separadas por un punto y coma.
Esto es muy útil con funciones que devuelvan múltiples valores.\\

Si una sentencia condicional \textit{if} no tiene ninguna condición, significa
\textit{true}, lo que en este contexto no es muy útil pero sí que lo es para
\textit{for} o \textit{switch}.\\

Veámoslo con un ejemplo:

%\begin{program}
\begin{verbatim} if x < 5 { f() } if x < 5 { f() } else if x == 5 { g() }
    
	if v := f(); v < 10 { fmt.Printf("\%d es menor que 10\n", v) } else
	{ fmt.Printf("\%d no es menor que 10", v) }
    
	// Se recogen los valores, y se comprueba // que err sea un valor no nulo
    
	if n, err = fd.Write(buffer); err != nil { ... } \end{verbatim}
%\caption{Sentencias condicionales if correctas} \end{program}

%\begin{program}
\begin{verbatim} // El else debe estar en la misma línea que // la llave de
cierre del if if v := f(); v < 10 { fmt.Printf("\%d es menor que 10\n", v)
} else { fmt.Printf("\%d no es menor que 10", v) } \end{verbatim}
%\caption{Sentencia condicionales if incorrectas} \end{program}

\section{Sentencia condicional: switch}

El \textit{switch} tiene la misma apariencia que en la mayoría de los lenguajes
y, aunque es en apariencia similar al de C, tiene varias diferencias:

\begin{itemize} \item Las expresiones no tienen por qué ser constantes ni
siquiera números.  \item Las condiciones no se evalúan en cascada
automáticamente.  \item Para conseguir evaluación en cascada, se puede usar al
final de cada \textit{case} la palabra reservada \textit{fallthrough}
finalizando la sentencia con un punto y coma.  \item Múltiples casos que
ejecuten una misma secuencia de instrucciones, pueden ser separados por comas.
\end{itemize}

%\begin{program}
\begin{verbatim} switch count%7 { case 4, 5, 6: error() case 3: a *= v;
fallthrough case 2: a *= v; fallthrough case 1: a *= v; fallthrough case 0:
return a*v } \end{verbatim}
%\caption{Sentencia condicional switch - Ejemplo de fallthrough} \end{program}

El \textit{switch} de Go es más potente que el de C. Además de que las
expresiones pueden ser de cualquier tipo, sabemos que la ausencia de la misma
significa \textit{true}. Así, podemos conseguir una cadena
\textit{if}$-$\textit{else} con un switch. Veamos otro ejemplo:

%\begin{program}
\begin{verbatim} // Forma habitual switch a { case 0: fmt.Printf("0") default:
fmt.Printf("No es cero") }
    
    
	// If-else a, b := x[i], y[j]; switch { case a < b: return -1 case a == b:
	return 0 case a > b: return 1 }
    
	// Este último switch sería equivalente a switch a, b := x[i], y[j]; { ...
	} \end{verbatim}
%\caption{Sentencia condicional switch} \end{program}

\section{Bucles: for}

Un bucle es una sentencia de los lenguajes de programación que permite ejecutar
un mismo grupo de instrucciones \textit{n} veces de forma consecutiva.\\

En Go sólo tenemos un tipo de bucles, a diferencia de otros lenguajes que suelen
poseer varios. Todo lo que implique un bucle, en Go se ejecuta con una
instrucción \textit{for}. Una de las cosas más llamativas de los bucles en Go,
es la posibilidad de usar asignaciones a varias variables, y hacer un bucle
doble en una sola instrucción. Veamos algunos ejemplos:

%\begin{program}
\begin{verbatim} // Forma clásica for i := 0; i < 10; i++ { // ejecutamos
instrucciones }
    
	// La falta de condición implica true (bucle infinito) for
	; ; { fmt.Printf("Bucle infinito") } for { fmt.Printf("Bucle infinito, sin
	poner ;") }
    
	// Equivalente al bucle while for a > 5 { fmt.Printf("Bucle while\n") a--
	} \end{verbatim}
%\caption{Bucles: for} \end{program}

%\begin{program}
\begin{verbatim} // Bucle doble for i,j := 0,N; i < j; i,j = i+1,j-1 { //
ejecutamos instrucciones }
    
	// Sería equivalente a for i := 0; i < N/2; i++{ for j := N; j > i; j--{ //
	ejecutamos instrucciones } } \end{verbatim}
%\caption{Bucles: for multivariable} \end{program}

\section{Instrucciones break y continue}

Las instrucciones \textit{break} y \textit{continue} nos permiten controlar la
ejecución de un bucle o una sentencia condicional, parándolo en un determinado
momento, saliendo del mismo y continuando la ejecución del programa, o saltando
a la siguiente iteración del bucle, respectivamente.\\

Funcionan de la misma forma que en C y se puede especificar una etiqueta para
que afecte a una estructura externa a donde es llamado. Como se podría pensar al
haber una etiqueta, sí, existe la sentencia \textit{goto}.

%\begin{program}
\begin{verbatim} Bucle: for i := 0; i < 10; i++ { switch f(i) { case 0, 1, 2:
break Loop } g(i) } \end{verbatim}
%\caption{Ejemplo de break} \end{program}

\section{Funciones\label{funciones}}

Las funciones nos permiten ejecutar un fragmento de código que realiza una
acción específica, de forma separada de un programa, que es donde se integra.\\

La principal ventaja del uso de funciones es la reutilización de código. Si
vamos a realizar la misma tarea un gran número de veces, nos evitamos la
duplicidad del código, y así la tarea de mantenimiento es mucho más sencilla.\\

Las funciones en Go comienzan con la palabra reservada \textit{func}. Después,
nos encontramos el nombre de la función, y una lista de parámetros entre
paréntesis. El valor devuelto por una función, si es que devuelve algún valor,
se encuentra después de los parámetros. Finalmente, utiliza la instrucción
\textit{return} para devolver un valor del tipo indicado.\\

Si la función devuelve varios valores, en la declaración de la función los tipos
de los valores devueltos van en una lista separada por comas entre paréntesis.

%\begin{program}
\begin{verbatim} func square(f float) float { return f*f }
    
	// Función con múltiples valores devueltos func MySqrt(f float) (float,
	bool) { if f >= 0 { return math.Sqrt(f), true } return 0, false
	}    \end{verbatim}
%\caption{Ejemplos de funciones} \end{program}

Los parámetros devueltos por una función son, en realidad, variables que pueden
ser usadas si se las otorga un nombre en la declaración. Además, están por
defecto inicializadas a ''cero'' (0, 0.0, false... dependiendo del tipo de la
variable).

%\begin{program}
\begin{verbatim} func MySqrt(f float) (v float, ok bool) { if f >= 0 { v, ok
= math.Sqrt(f), true } else { v, ok = 0, false } return v, ok }    
    
	// Función con múltiples valores devueltos func MySqrt(f float) (v float, ok
	bool) { if f >= 0 { v, ok = math.Sqrt(f), true } return v, ok
	}    \end{verbatim}
%\caption{Ejemplos de funciones con variables de retorno} \end{program}

Finalmente, un \textit{return} vacío devuelve el actual valor de las variables
de retorno. Vamos a ver dos ejemplos más de la función \textit{MySqrt}:

%\begin{program}
\begin{verbatim} func MySqrt(f float) (v float, ok bool) { if f >= 0 { v, ok
= math.Sqrt(f), true } return           // Debe ser explícito }    
    
	// Función con múltiples valores devueltos func MySqrt(f float) (v float, ok
	bool) { if f < 0 { return       // Caso de error } return math.Sqrt(f), true
	}    \end{verbatim}
%\caption{Ejemplos de funciones con return vacío} \end{program}

\section{¿Y qué hay del 0?}

En esta sección vamos a tratar el tema de inicialización de las variables a un
valor estándar ``cero''. En Go, por temas de seguridad, toda la memoria
utilizada es previamente inicializada. Todas las variables son inicializadas con
su valor asignado o con un valor ``cero'' en el momento de su declaración. En
caso de inicializar una variable con el valor ``cero", se usará el valor tomado
como ``cero'' para su tipo concreto.\\

Como los valores dependerán del tipo, veamos cuáles son esos posibles valores:

\begin{itemize} \item \textbf{int:} 0 \item \textbf{float:} 0.0 \item
\textbf{bool:} false \item \textbf{string:} ``'' \item \textbf{punteros:} nil
\item \textbf{struct:} valores ``cero'' de todos sus atributos.  \end{itemize}

Por ejemplo, el siguiente bucle imprimirá 0, 0, 0, 0, 0. Esto se debe a que la
variable v se declara en cada pasada del bucle, con lo que se reinicializa.

%\begin{program}
\begin{verbatim} for i := 0; i < 5; i++ { var v int fmt.Printf("\%d\n", v)
v = 5 } \end{verbatim}
%\caption{Ejemplos de valores ``cero"} \end{program}

\section{Defer}

La instrucción \textit{defer} es una instrucción poco común en lenguajes de
programación, pero que puede resultar muy útil. En el caso de Go, la instrucción
\textit{defer} ejecuta una determinada función o método cuando la función que la
engloba termina su ejecución.\\

La evaluación de los argumentos de \textit{defer} se realiza en el momento de la
llamada a dicha instrucción, mientras que la llamada a la función no ocurre
hasta que se ejecuta el \textit{return} correspondiente.\\

Es una instrucción muy útil para cerrar descriptores de ficheros o desbloquear
mutex. Veamos un pequeño ejemplo, en el que cerraremos un fichero al terminar la
ejecución de la función.

%\begin{program}
\begin{verbatim} func data(name string) string { f := os.Open(name, os.O_RDONLY,
0) defer f.Close() contenido := io.ReadAll(f) return contenido } \end{verbatim}
%\caption{Ejemplo de defer} \end{program}

Cada \textit{defer} se corresponde con una llamada a una única función. Cada vez
que se ejecuta un \textit{defer} una llamada a la función que deba ser ejecutada
se encola en una pila LIFO, de tal forma que la última llamada será la primera
en ejecutarse. Con un bucle, podemos cerrar todos los ficheros o desbloquear los
mutex de un golpe al final. Veamos un ejemplo, que en este caso imprimirá los
números 4, 3, 2, 1 y 0 en dicho orden.

%\begin{program}
\begin{verbatim} func f(){ for i := 0; i < 5; i++ { defer fmt.Printf("\%d", i)
} } \end{verbatim}
%\caption{Ejemplos de varias ejecuciones de funciones con defer} \end{program}

Veamos un ejemplo más completo para entender correctamente el funcionamiento de
\textit{defer}. Para ello, vamos a seguir la traza de ejecución de nuestro
programa, imprimiendo la secuencia de funciones que son llamadas.

%\begin{program}
\begin{verbatim} func trace (s string) { fmt.Print("entrando en: ", s, "\n") }

	func untrace (s string) { fmt.Print("saliendo de: ", s, "\n") }

	func a() { trace("a") defer untrace("a") fmt.Print("estoy en a\n") }

	func b() { trace("b") defer untrace("b") fmt.Print("estoy en b\n") a() }

	func main() { b() } \end{verbatim}
%\caption{Siguiendo la traza de ejecución con defer: I} \end{program}

El programa anterior, imprimirá los siguientes valores por pantalla:

\begin{verbatim} entrando en: b estoy en b entrando en: a estoy en a saliendo
de: a saliendo de: b \end{verbatim}

Veamos ahora una versión más optimizada del mismo programa, que imprimirá
exactamente los mismos valores por pantalla. En este ejemplo se puede observar
que los argumentos de \textit{defer} son evaluados en el momento, pero que la
ejecución de la función no se realiza hasta finalizar su función ``padre''.
\clearpage
%\begin{program}
\begin{verbatim} func trace (s string) string { fmt.Print("entrando en: ", s,
"\n") return s }

	func un(s string) { fmt.Print("saliendo de: ", s, "\n") }

	func a() { defer un(trace("a")) fmt.Print("estoy en a\n") }

	func b() { defer un(trace("b")) fmt.Print("estoy en b\n") a() }

	func main() { b() } \end{verbatim}
%\caption{Siguiendo la traza de ejecución con defer: II} \end{program}

\section{Funciones anónimas, $\lambda$-programación y closures}

Los tres conceptos están muy relacionados entre sí. Las funciones anónimas y la
$\lambda$-programación se refieren al mismo concepto, mientras que las
\textit{closures} son un uso concreto de las funciones anónimas.

\begin{description} \item[Funciones anónimas o $\lambda$-programación] Se basa
en utilizar funciones que están definidas en tiempo de ejecución. Este tipo de
funciones, no tienen nombres por sí mismas y se ejecutan en el contexto en el
que son declaradas. A menudo, son referenciadas por variable para poder ser
ejecutadas en otro contexto. Son típicas de lenguajes funcionales como, por
ejemplo, Haskell.  \item[Closures] Se pueden definir las \textit{closures} como
funciones anónimas declaradas en el ámbito de otra función y que pueden hacer
referencia a las variables de su función contenedora incluso después de que esta
se haya terminado de ejecutar.  \end{description}

Dado que son dos conceptos bastante abstractos, lo mejor es que veamos un
ejemplo de cada uno de ellos.

%\begin{program}
\begin{verbatim} func f(){ for i := 0; i < 10; i++{ g := func(i int)
{ fmt.Printf("\%d", i) } g(i) } } \end{verbatim}
%\caption{Funciones anónimas o $\lambda$-programación} \end{program}

%\begin{program}
\begin{verbatim} func sumador(){ var x int return func(delta int) int{ x +=
delta return x } }
    
	var f = sumador()    // f es una función ahora.  fmt.Printf(f(1))
	fmt.Printf(f(20)) fmt.Printf(f(300)) \end{verbatim}
%\caption{Ejemplo de closures} \end{program}

El anterior ejemplo, imprimirá los valores 1, 21 y 321, ya que irá acumulando
los valores en la variable \textit{x} de la función \textit{f}.

\clearpage \thispagestyle{empty} \ \\ \newpage
