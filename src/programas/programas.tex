
\chapter{Estructuración de programas}

\section{Paquetes y estructura de los ficheros fuente}

Todos los programas en Go están construidos como paquetes, que pueden usar
funciones de otros paquetes. Un programa está por lo tanto creado mediante el
enlace de un conjunto de paquetes y cada paquete, se forma de uno o varios
ficheros fuente.\\

La importación de paquetes en Go se realiza de forma ``calificada", es decir,
que cuando nos referimos al ítem de un paquete, hay que hacer una referencia
explícita al nombre del paquete en cuestión. Así pues, tendremos una sintaxis
similar a: \textit{nombreDelPaquete.NombreDelItem}.\\

Todo fichero fuente contiene:

\begin{itemize} \item Una cláusula de instanciación del paquete. Es decir, a qué
paquete pertenece dicho fichero. El nombre usado es el nombre utilizado por
defecto por los paquetes que lo importen.  \begin{verbatim} package fmt
\end{verbatim} \item Un conjunto opcional de declaraciones de importación de
otros paquetes.  \begin{verbatim} import "fmt"         // Usa el nombre por
defecto import my_fmt "fmt"  // Usa en este fichero el nombre my_fmt
\end{verbatim} \item Puede existir alguna declaración global o declaración
a ``nivel de paquete".
	%\begin{program}
	\begin{verbatim} package main     // Este fichero forma parte del paquete
	"main"
    
	   import "fmt"     // Este fichero utiliza el paquete "fmt"
    
	   const hola = "Hello, World!\n"
    
	   func main() { fmt.Print(hola)  // fmt es el nombre del paquete importado
	   } \end{verbatim}
	%\caption{Ejemplo de un paquete con un único fichero fuente} \end{program}
\end{itemize}



\section{Ámbito de una variable}

El ámbito de una variable define la visibilidad de la misma por otras funciones,
métodos, o incluso por otros paquetes, y que ésta pueda ser usada o no fuera de
su entorno local.\\

Dentro de un paquete todas las variables globales, funciones, tipos y constantes
son visibles desde todos los ficheros que formen dicho paquete.\\

Para los usuarios de dicho paquete - otros paquetes que lo importen -, los
nombres deben comenzar por una letra mayúscula para que sean visibles. Esto se
acepta para variables globales, funciones, tipos, constantes, además de métodos
y atributos de una estructura, aplicados a variables globales y tipos. Su
equivalente en C/C++ serían los modificadores: \textit{extern}, \textit{static},
\textit{private} y \textit{public}.

%\begin{program}
\begin{verbatim} const hola = "Hola"         // Variable visible en el paquete
const Hola = "Ey, Hola!"    // Variable visible globalmente const _Adios = "Ciao
ciao"  // _ no es una letra mayúscula \end{verbatim}
%\caption{Ejemplo de visibilidad} \end{program}

\section{Inicialización}

Todos los programas en Go deben contener un paquete llamado \textit{main} y tras
la etapa de inicialización, la ejecución del programa comienza por la función
\textit{main()} de dicho paquete, de forma análoga a la función global
\textit{main()} en C o C++.\\

La función \textit{main.main()} no acepta argumentos y no devuelve ningún valor.
El programa finaliza su ejecución de forma inmediata y satisfactoriamente cuando
la función \textit{main.main()} termina.\\

Como se ha mencionado en el primer párrafo, antes de comenzar la ejecución
existe una etapa de inicialización. Esta etapa se encarga de inicializar las
variables globales antes de la ejecución de \textit{main.main()}. Existen dos
formas de hacer esta inicialización:

\begin{enumerate} \item Una declaración global con un inicializador.  \item Una
función \textit{init()}. Pueden existir tantas como una por fichero fuente.
\end{enumerate}

La dependencia entre los paquetes garantiza un orden de ejecución correcto. La
inicialización siempre se realiza de forma ``mono-hilo", es decir, no existe
ningún tipo de concurrencia en la etapa de inicialización para hacerla más
segura. Veamos un ejemplo de inicialización:

%\begin{program}
\begin{verbatim} package trascendental import "math" var Pi float64
    
    
    
	func init() { Pi = 4 * math.Atan(1)   // init() calcula Pi }
    
	---------------------
    
	package main import ( "fmt"; "trascendental")
    
	// La declaración calcula dosPi var dosPi = 2 * trascendental.Pi
    
	func main() { fmt.Printf("2*Pi = \%g\n", dosPi) } \end{verbatim}
%\caption{Ejemplo de inicialización} \end{program}

%%%%%%%%%%%%%%%%%%%%%%%
% ¿Hablamos de compilación de paquetes? %
%%%%%%%%%%%%%%%%%%%%%%%

\section{Testing}

Go provee al programador de una serie de recursos muy interesantes. Entre ellos,
cabe destacar el sistema de pruebas que ofrece al programador para generar las
pruebas de su código y realizar un test automático.\\

Para probar un paquete hay que escribir un conjunto de ficheros fuente dentro
del mismo paquete, con nombres del tipo \textit{*\textunderscore test.go}.\\

Dentro de esos ficheros se declararán las funciones globales que le darán
soporte a la herramienta de test \textit{gotest}. Las funciones deben seguir el
siguiente patrón: Test[\textasciicircum a-z]*, es decir, la palabra ``Test"
seguida de un número indefinido de letras. Además deben de tener la declaración
de la siguiente forma:

\begin{verbatim} func TestXxxx (t *testing.T) \end{verbatim}

El paquete \textit{testing} proporciona una serie de utilidades de
\textit{logging} y \textit{error reporting}.\\

Veamos un ejemplo sacado de la librería \textit{fmt} de Go:

%\begin{program}
\begin{verbatim} package fmt import ( "testing")
    
    
    
	func TestFlagParser(t *testing.T) { var flagprinter flagPrinter for _, tt :=
	range flagtests { s := Sprintf(tt.in, &flagprinter) if s != tt.out
	{ t.Errorf("Sprintf(\%q, &flagprinter) => \%q, want \%q", tt.in, s, tt.out)
	} } } \end{verbatim}
%\caption{Ejemplo de testing: fmt\textunderscore test.go} \end{program}

Al utilizar la herramienta \textit{gotest} se ejecutan todos los ficheros
*\textunderscore test.go. Al ejecutarlo sobre el fichero anteriormente nombrado
\textit{fmt\textunderscore test.go}, obtenemos la siguiente salida:

\begin{verbatim} mallocs per Sprintf(""): 2 mallocs per Sprintf("xxx"):
3 mallocs per Sprintf("\%x"): 5 mallocs per Sprintf("\%x \%x"): 7 PASS
\end{verbatim}

\section{Librerías}

Las librerías son paquetes normales pero que por convención se suele llamar así
a los paquetes que vienen integrados dentro del propio lenguaje.\\

Actualmente las librerías existentes son pocas, pero están creciendo a un ritmo
vertiginoso. Veamos en una tabla algunos ejemplos:

\begin{table}[htb] \begin{center} \begin{tabular}{|c|c|c|} \hline
\textbf{Paquete} & \textbf{Propósito} & \textbf{Ejemplos}\\ \hline fmt & E/S
formateada & Printf, Sprintf...\\ os & Interfaz del SO & Open, Read, Write...\\
strconv & números $<->$ string & Atoi, Atof, Itoa...\\ io & E/S genérica & Copy,
Pipe...\\ flag & flags: --help, etc. & Bool, String...\\ log & Log de eventos
& Log, Logf, Stderr...\\ regexp & Expresiones regulares & Compile, Match...\\
template & HTML, etc. & Parse, Execute...\\ bytes & byte arrays & Compare,
Buffer...\\ \hline \end{tabular} \end{center} \caption{Tabla de paquetes}
\end{table}

	\subsection{Fmt}
	
	El paquete \textit{fmt} implementa las funciones de entrada / salida
	formateada, de manera análoga a otras funciones en otros lenguajes como C.
	
		\subsubsection{Modificadores}
	
		Las funciones que utilizan utilizan un formato para imprimir los datos,
		usan una serie de modificadores para imprimir los distintos tipos de
		datos. A continuación hay una lista de los modificadores existentes:
	
		\begin{itemize} \item \textbf{General:} \begin{itemize} \item \%v
		imprime el valor en un formato por defecto.\\ Cuando se imprimen
		structs, el flag + (\%+v) añade los nombres de los campos.  \item \%\#v
		una representación en ``sintaxis-Go'' del valor \item \%T  una
		representación en ``sintaxis-Go'' del tipo del valor \end{itemize} \item
		\textbf{Boolean:} \begin{itemize} \item \%t  la palabra true o false
		\end{itemize} \item \textbf{Enteros:} \begin{itemize} \item \%b	base
		2 \item \%c	el caracter representado por su código Unicode \item \%d
		base 10 \item \%o	base 8 \item \%x	base 16, con letras minúsculas
		de a-f \item \%X	base 16, con letras mayúsculas de A-F \end{itemize}
		\item \textbf{Reales:} \begin{itemize} \item \%e  notación científica,
		p.ej. -1234.456e+78 \item \%E  notación científica, p.ej. -1234.456E+78
		\item \%f  coma flotante sin exponente, e.g. 123.456 \item \%g
		cualquiera que \%e o \%f produzca pero de manera más compacta \item \%G
		cualquiera que \%E o \%f produzca pero de manera más compacta
		\end{itemize} \item \textbf{Cadenas de caracteres o slices de bytes:}
		\begin{itemize} \item \%s  los bytes sin interpretar de un string
		o slice \item \%q  una cadena de caracteres con dobles comillas escapada
		de forma segura con una sintaxis Go \item \%x  notación base 16 con dos
		caracteres por byte \end{itemize} \item \textbf{Punteros:}
		\begin{itemize} \item \%p	notación base 16, con un 0x precedente
		\end{itemize}

			\item No hay flag 'u'. Los enteros se imprimen sin signo si son del
			tipo unsigned.  \end{itemize}
	
	\clearpage
	
		\subsubsection{Fprintf}
		
		\begin{verbatim} func Fprintf(w io.Writer, format string,
		a ...interface{}) (n int, error os.Error) \end{verbatim}
		
		La función \textit{Fprintf} escribe en un elemento de tipo
		\textit{io.Writer} (es decir, un descriptor de fichero o similares), con
		el formato indicado en el parámetro \textit{format}. El resto de
		parámetros son las variables en cuestión que deben imprimirse.\\
		
		\textit{Fprintf} devuelve 2 valores: \begin{itemize} \item \textbf{n}:
		Indica el número de caracteres escritos correctamente.  \item
		\textbf{error}: Contiene un valor significativo si ha habido algún
		error.  \end{itemize}
		
		Veamos un ejemplo de cómo se usaría la función:
		
		\begin{verbatim} var x int = 2 var punt *int
		
			fmt.Fprintf (os.Stdout, "La variable x que vale \%d, está en la
			posición \%p", x, &punt) \end{verbatim}
		
		\subsubsection{Printf}
		
		\begin{verbatim} func Printf(format string, a ...interface{}) (n int,
		errno os.Error) \end{verbatim}
		
		La función \textit{Printf} escribe por la salida estándar con el formato
		indicado en el parámetro \textit{format}. El resto de parámetros son las
		variables en cuestión que deben imprimirse. Realmente, \textit{Printf}
		realiza una llamada a \textit{Fprintf} con el primer parámetro igual
		a \textit{os.Stdout}.\\
		
		\textit{Printf} devuelve 2 valores: \begin{itemize} \item \textbf{n}:
		Indica el número de caracteres escritos correctamente.  \item
		\textbf{errno}: Contiene un valor significativo si ha habido algún
		error.  \end{itemize}
		
		Veamos un ejemplo de cómo se usaría la función:
		
		\begin{verbatim} var x int = 2 var punt *int
		
			fmt.Printf ("La variable x que vale \%d, está en la posición \%p",
			x, &punt) \end{verbatim}
		
		\subsubsection{Sprintf}
		
		\begin{verbatim} func Sprintf(format string, a ...interface{}) string
		\end{verbatim}
		
		La función \textit{Sprintf} escribe en un buffer intermedio la cadena de
		caracteres pasada con el formato indicado en el parámetro
		\textit{format}. El resto de parámetros son las variables en cuestión
		que deben imprimirse.\\
		
		\textit{Sprintf} devuelve la cadena de caracteres que hemos
		formateado.\\
		
		Veamos un ejemplo de cómo se usaría la función:
		
		\begin{verbatim} var x int = 2 var punt *int var s string
		
			s = fmt.Sprintf ("La variable x que vale \%d, está en la posición
			\%p", x, &punt) \end{verbatim}
		
		\subsubsection{Fprint, Print y Sprint}
		
		\begin{verbatim} func Fprint(w io.Writer, a ...interface{}) (n int,
		error os.Error) func Print(a ...interface{}) (n int, errno os.Error)
		func Sprint(a ...interface{}) string \end{verbatim}
		
		Las funciones \textit{Fprint}, \textit{Print} y \textit{Sprint} son
		generalizaciones de las funciones anteriormente descritas. Como se puede
		observar, cambian el nombre eliminando la 'f' final y el parámetro que
		se corresponde con el formato. Así pues, estas funciones imprimen en el
		formato por defecto que el lenguaje considere.\\
				
		Veamos un ejemplo de cómo se usarían las funciones:
		
		\begin{verbatim} fmt.Fprint (os.Stderr, "Error: 12") fmt.Print ("ACM da
		muchos cursos geniales\n") s = fmt.Sprint ("Go mola un montón")
		\end{verbatim}
		
		La salida del programa sería:
		
		\begin{verbatim} Error: 12ACM da muchos cursos geniales Go mola un
		montón \end{verbatim}
		
		\subsubsection{Fprintln, Println y Sprintln}
		
		\begin{verbatim} func Fprintln(w io.Writer, a ...interface{}) (n int,
		error os.Error) func Println(a ...interface{}) (n int, errno os.Error)
		func Sprintln(a ...interface{}) string \end{verbatim}
		
		Las funciones \textit{Fprintln}, \textit{Println} y \textit{Sprintln}
		son generalizaciones de las funciones con formato anteriormente
		descritas. Como se puede observar, estas funciones terminan en 'ln' lo
		que indica que al final del último operando que se le pase a la función,
		añaden un salto de línea automáticamente.\\
				
		Veamos un ejemplo de cómo se usarían las funciones:
		
		\begin{verbatim} fmt.Fprintln (os.Stderr, "Error: 12") fmt.Println ("ACM
		da muchos cursos geniales") s = fmt.Sprintln ("Go mola un montón")
		\end{verbatim}
		
		La salida del programa sería:
		
		\begin{verbatim} Error: 12 ACM da muchos cursos geniales Go mola un
		montón \end{verbatim}

	\subsection{Os}
	
	El paquete \textit{os} se compone de varios ficheros que tratan toda la
	comunicación con el Sistema Operativo. Vamos a ver algunas de las funciones
	más utilizadas de dicho paquete y que están repartidas en diversos ficheros.
	
		\subsubsection{ForkExec}

		\begin{verbatim} func ForkExec(argv0 string, argv []string, envv
		[]string, dir string, fd []*File) (pid int, err Error) \end{verbatim}
				
		La función \textit{ForkExec} crea un nuevo hilo de ejecución dentro del
		proceso actual e invoca al comando Exec (ejecuta otro programa) con el
		programa indicado en \textit{argv0}, con los argumentos \textit{argv[]}
		y el entorno descrito en \textit{envv}.\\
		
		El array \textit{fd} especifica los descriptores de fichero que se
		usarán en el nuevo proceso: fd[0] será el descriptor 0 (equivalente
		a os.Stdin o entrada estándar), fd[1] será el descriptor 1 y así
		sucesivamente. Pasándole un valor de \textit{nil} hará que el proceso
		hijo no tenga los descriptores abiertos con esos índices.\\
		
		Si \textit{dir} no está vacío, el proceso creado se introduce en dicho
		directorio antes de ejecutar el programa.\\
		
		\textit{ForkExec} devuelve 2 valores: \begin{itemize} \item
		\textbf{pid}: Indica el id de proceso (pid) del proceso hijo.  \item
		\textbf{err}: Contiene un valor significativo si ha habido algún error.
		\end{itemize}
		
		Veamos un ejemplo de cómo se usaría la función:
		
		\begin{verbatim} var argv []string { "/bin/ls", "-l" } var env []string
		
			pid, ok := os.ForkExec ("/bin/ls", argv, env, "/home", null)
			\end{verbatim}
		
		\subsubsection{Exec}

		\begin{verbatim} func Exec(argv0 string, argv []string, envv []string)
		Error \end{verbatim}
				
		La función \textit{Exec} sustituye el proceso actual pasando a ejecutar
		otro programa mediante el comando Exec, que ejecuta otro programa
		indicado en \textit{argv0}, con los argumentos \textit{argv[]} y el
		entorno descrito en \textit{envv}.\\
		
		Si el comando se ejecuta correctamente, nunca retorna al programa
		principal. En caso de fallo devuelve un error.\\
		
		ForkExec es, casi siempre, una mejor opción para ejecutar un programa,
		ya que tras su ejecución volvemos a tener control sobre el código
		ejecutado.\\
		
		\textit{Exec} devuelve un error en caso de que el comando nose ejecute
		correctamente.\\
		
		Veamos un ejemplo de cómo se usaría la función:
		
		\begin{verbatim} var argv []string { "/bin/ls", "-l" } var env []string
		
			error := os.Exec ("/bin/ls", argv, env) \end{verbatim}
		
		\subsubsection{Wait}

		\begin{verbatim} func Wait(pid int, options int) (w *Waitmsg, err Error)
		\end{verbatim}
				
		La función \textit{Wait} espera a que termine o se detenga la ejecución
		de un proceso con identificador \textit{pid} y retorna un mensaje
		describiendo su estado y un error, si es que lo hay. El parámetro
		\textit{options} describe las opciones referentes a la llamada
		\textit{wait}.\\
		
		\textit{Wait} devuelve 2 valores: \begin{itemize} \item \textbf{w}: Un
		mensaje con el estado del proceso.  \item \textbf{err}: Contiene un
		valor significativo si ha habido algún error.  \end{itemize}
		
		Veamos un ejemplo de cómo se usaría la función:
		
		\begin{verbatim} var argv []string { "/bin/ls", "-l" } var env []string
		
		   if pid, ok := os.ForkExec ("/bin/ls", argv, env, "/home", nil); ok
		   { msg, ok := os.Wait(pid, 0) }	    \end{verbatim} \clearpage
		   \subsubsection{Getpid}

		\begin{verbatim} func Getpid() int \end{verbatim}
				
		La función \textit{Getpid} devuelve el identificador del proceso que
		ejecuta la función. 
		
		\subsubsection{Getppid}

		\begin{verbatim} func Getppid() int \end{verbatim}
				
		La función \textit{Getppid} devuelve el identificador del proceso padre
		del que ejecuta la función. 
		
		\subsubsection{Exit}

		\begin{verbatim} func Exit(code int) \end{verbatim}
				
		La función \textit{Exit} termina la ejecución del programa con un código
		de error indicado en el parámetro \textit{code}.
		
	\subsection{Os - Entrada/salida en ficheros}
	
	La escritura en ficheros depende del paquete \textit{os}. A pesar de ello,
	se ha creado una nueva sección en el manual para tratar las funciones de
	tratamiento de ficheros para una mejor comprensión de las mismas.
	
		\subsubsection{El tipo File}
		
		Los ficheros en un Sistema Operativo son identificados a través de
		identificadores de ficheros, que indican con un número cada fichero
		abierto en el sistema y accesible por cada proceso. El tipo
		\textit{File} representa un descriptor de fichero abierto que trata como
		si fuera un objeto, y está definido dentro del paquete \textit{os} como:
		
		\begin{verbatim} // File represents an open file descriptor.  type File
		struct { fd      int name    string dirinfo *dirInfo // nil a menos que
		sea un directorio nepipe  int } \end{verbatim}
		
		Los parámetros importantes son los dos primeros:
		
		\begin{itemize} \item \textbf{fd}: Indica el descriptor de fichero del
		archivo.  \item \textbf{name}: Indica el nombre relativo del fichero
		(hello.go y no /home/yo/hello.go) \end{itemize}
		
		Echemos ahora un vistazo a las funciones que nos permiten manejar
		ficheros.
		
		\subsubsection{Fd}

		\begin{verbatim} func (file *File) Fd() int \end{verbatim}
				
		La función \textit{Fd} devuelve el descriptor de fichero del archivo
		\textit{file} que llama a la función.\\
		
		\textbf{Nota.-} Nótese que Fd no recibe ningún parámetro, sino que es un
		objeto quien invocará a dicho método. Para una mejor comprensión se
		recomienda repasar este punto tras la lectura del capítulo 6.
		
		\subsubsection{Name}

		\begin{verbatim} func (file *File) Name() string \end{verbatim}
				
		La función \textit{Name} devuelve el nombre relativo del fichero
		\textit{file} que llama a la función.
		
		\subsubsection{Open}
		
		Las función \textit{Open} que vamos a ver a continuación utiliza una
		serie de modificadores para abrir los ficheros en distintos modos. Los
		modificadores existentes son:
		
		\begin{verbatim} O_RDONLY  // Abre el fichero en modo sólo lectura
		O_WRONLY  // Abre el fichero en modo sólo escritura O_RDWR  // Abre el
		fichero en modo lectura-escritura O_APPEND  // Abre el fichero en modo
		append (escribir al final) O_ASYNC  // Genera una señal cuando la E/S
		está disponible O_CREAT  // Crea un nuevo fichero si no existe O_EXCL
		// Usado con O_CREAT. El fichero debe no existir O_SYNC  // Abre el
		fichero listo para E/S síncrona O_TRUNC  // Si es posible trunca el
		fichero al abrirlo O_CREATE  // Igual a O_CREAT \end{verbatim}
		
		La función \textit{Open} tiene el siguiente formato:
		
		\begin{verbatim} func Open(name string, flag int, perm int) (file *File,
		err Error) \end{verbatim}
		
		La función abre el fichero indicado (con su ruta absoluta o relativa) en
		el parámetro \textit{name}, del modo que se especifica en el parámetro
		\textit{flag} usando los flags anteriormente descritos, con los permisos
		indicados en el último parámetro.\\
		
		\textit{Open} devuelve 2 valores: \begin{itemize} \item \textbf{file}:
		Un puntero a un objeto de tipo *File, que usaremos después para llamar
		a los distintos métodos.  \item \textbf{err}: Un error en caso de que la
		apertura no haya funcionado correctamente.  \end{itemize} \clearpage
		Veamos un ejemplo de cómo se usaría la función:
		
		\begin{verbatim} var fichero *os.File var error os.Error
		    
			fichero, error = os.Open("/home/yo/mifichero", O_CREAT | O_APPEND,
			660) \end{verbatim}
		
		\subsubsection{Close}
		
		\begin{verbatim} func (file *File) Close() Error \end{verbatim}
		
		La función \textit{Close} cierra el fichero que invoca dicho método.\\
		
		\textit{Close} devuelve un error en caso de que la llamada no haya
		podido completarse correctamente.\\
		
		Veamos un ejemplo de cómo se usaría la función:
		
		\begin{verbatim} var fichero *os.File var error os.Error
		    
			fichero, error = os.Open("/home/yo/mifichero", O_CREAT | O_APPEND,
			660) fichero.Close() \end{verbatim}
		
		\subsubsection{Read}
		
		\begin{verbatim} func (file *File) Read(b []byte) (n int, err Error)
		\end{verbatim}
		
		La función \textit{Read} lee caracteres hasta el número máximo de bytes
		que haya en el parámetro \textit{b}, que se consigue con la llamada
		\textit{len(b)}.\\
		
		\textit{Read} devuelve 2 valores: \begin{itemize} \item \textbf{n}: El
		número de bytes leídos realmente.  \item \textbf{err}: Contiene un valor
		significativo si ha habido algún error.  \end{itemize}
		
		Veamos un ejemplo de cómo se usaría la función:
		
		\begin{verbatim} var fichero *os.File var error os.Error var bytes
		[256]byte var num_leidos int
		    
			fichero, error = os.Open("/home/yo/mifichero", O_CREAT | O_RDONLY,
			660) num_leidos, error = fichero.Read (byte) fichero.Close()
			\end{verbatim}
		
		\subsubsection{Write}
		
		\begin{verbatim} func (file *File) Write(b []byte) (n int, err Error)
		\end{verbatim}
		
		La función \textit{Write} escribe caracteres hasta el número de bytes
		que haya en el parámetro \textit{b}.\\
		
		\textit{Write} devuelve 2 valores: \begin{itemize} \item \textbf{n}: El
		número de bytes escritos realmente.  \item \textbf{err}: Contiene un
		valor significativo si ha habido algún error.  \end{itemize}
		
		Veamos un ejemplo de cómo se usaría la función:
		
		\begin{verbatim} var fichero *os.File var error os.Error var bytes
		[256]byte = { 'H', 'o', 'l', 'a' } var num_leidos int
		    
			fichero, error = os.Open("/home/yo/mifichero", O_CREAT | O_APPEND,
			660) num_leidos, error = fichero.Write (byte) fichero.Close()
			\end{verbatim}
		
		\subsubsection{WriteString}
		
		\begin{verbatim} func (file *File) WriteString(s string) (ret int, err
		Error) \end{verbatim}
		
		La función \textit{WriteString} escribe una cadena de caracteres
		(string). Es mucho más cómodo usar esta función para escribir datos en
		un fichero.\\
		
		\textit{WriteString} devuelve 2 valores: \begin{itemize} \item
		\textbf{ret}: El número de caracteres escritos realmente.  \item
		\textbf{err}: Contiene un valor significativo si ha habido algún error.
		\end{itemize}
		
		Veamos un ejemplo de cómo se usaría la función:
		
		\begin{verbatim} var fichero *os.File var error os.Error var cadena
		String = "Hola" var num_leidos int
		    
			fichero, error = os.Open("/home/yo/mifichero", O_CREAT | O_APPEND,
			660) num_leidos, error = fichero.WriteString (cadena)
			fichero.Close() \end{verbatim}
		
	\subsection{Otras librerías}
	
	\textit{Go} es un lenguaje en desarrollo, y como tal sus librerías están
	todavía creciendo aunque lo hacen a pasos agigantados. Una de las mejores
	características de \textit{Go} es que tiene unas librerías muy bien
	documentadas, tanto en el código como en la documentación online de la
	página web oficial.\\
	
	A continuación se mencionan algunas de las librerías más comúnmente usadas,
	que el lector puede entender perfectamente al finalizar este curso.
	
	\begin{itemize} \item \textbf{bufio}: Es una librería con funciones
	encargadas de realizar E/S a través de un buffer.  \item \textbf{crypto}:
	Una interesante librería con múltiples funciones criptográficas.  \item
	\textbf{flag}: Esta librería se encarga de manejar todos los parámetros
	pasados al programa mediante la línea de comandos.  \item \textbf{http}: Una
	librería que contiene un servidor http y todas las funciones necesarias para
	implementar un programa para la web.  \item \textbf{image}: Contiene
	funciones para el manejo de imágenes, bitmaps...  \item \textbf{math}:
	Librería de funciones matemáticas de todo tipo.  \item \textbf{reflect}: La
	librería encargada de gestionar funciones con número variable de parámetros.
	(Se verá en detalle más adelante).  \item \textbf{strings}: Una serie de
	funciones para manejo de cadenas de caracteres.  \item \textbf{testing}: La
	librería más completa para realizar pruebas sobre tu propio programa.
	\end{itemize}
