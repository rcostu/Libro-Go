% Anexo: Herramientas de Go
\chapter{Herramientas de Go}

\section{Compiladores}

Go viene con un conjunto de herramientas bastante completo. Entre las
herramientas más importantes, están los dos tipos de compiladores que podemos
usar para generar nuestros programas ejecutables: \emph{gccgo y 6g/8g}.\\

Vamos a ver características de uno y otro, y un ejemplo de cómo realizar una
compilación con el compilador nativo de Go, la versión 6g/8g. Una característica
común de ambos, es que generan un código únicamente entre un 10\% y un 20\% más
lento que código en C.

	\subsection{gccgo}
	
	El compilador \emph{gccgo} es un \textit{front-end} del famoso compilador
	de C, GCC. Posee las siguientes características:
	
	\begin{itemize} \item Es un compilador más tradicional.  \item Soporta
	32-bit y 64-bit bajo x86, además de ARM.  \item Genera muy buen código, pero
	no tan rápido como su hermano \emph{6g/8g}.  \item Se puede enlazar con
	GCC, y así realizar una compilación con C.  \item No soporta pilas
	segmentadas todavía y aloja cada goroutine - se verá más adelante - por hilo
	de ejecución.  \end{itemize}
	
	Su uso, es exactamente igual al uso que se le da a GCC, sólo que invocándolo
	con el comando gccgo.
	
	\subsection{6g/8g}
	
	El compilador \emph{6g/8g} es un compilador experimental nativo de Go.
	\emph{6g}, es el compilador asociado a la arquitectura amd64, y genera
	ficheros objeto con extensión ".6". \emph{8g}, es el compilador asociado
	a la arquitectura 386, y genera ficheros objeto con extensión ".8".
	\clearpage \begin{itemize} \item Es un compilador experimental.  \item
	Soporta 32-bit y 64-bit bajo x86, además de ARM.  \item Genera buen código
	de forma muy, muy rápida.  \item No se puede enlazar con GCC, pero tiene
	soporte FFI.  \item Posee un buen soporte de gouroutines, multiplexándolas
	en varios hilos de ejecución, e implementa las pilas segmentadas.
	\end{itemize}
	
	Para compilar un archivo cualquiera llamado \textbf{file.go}, usaremos el
	comando:
	
	\begin{verbatim} $ 6g file.go \end{verbatim}
	
	y para enlazar el archivo y así generar el fichero ejecutable
	correspondiente, usaremos:
	
	\begin{verbatim} $ 6l file.6 \end{verbatim}
	
	Finalmente, para ejecutar el programa usaremos el comando:
	
	\begin{verbatim} $ ./6.out \end{verbatim}
	
	\textbf{Nota.-} Hay que tener en cuenta, que si se usa la versión 32-bit del
	compilador, se cambiaría cada 6 por un 8.\\
	
	\textbf{Nota.-} Para conseguir compilar un fichero con su propio nombre (y
	así no usar el fichero ejecutable por defecto (6.out), podemos pasarle el
	parámetro \emph{-o fichero\textunderscore salida} (ejemplo: 6l -o fichero
	file.6).\\
	
	El \emph{linker} de Go (\textit{6l}), no necesita recibir ningún otro
	fichero del que dependa la compilación, como en otros lenguajes, ya que el
	compilador averigua qué ficheros son necesarios leyendo el comienzo del
	fichero compilado.\\
	
	A la hora de compilar un fichero \textbf{A.go} que dependa de otro
	\textbf{B.go} que depende a su vez de \textbf{C.go}:
	
	\begin{itemize} \renewcommand{\labelitemi}{-} \item Compila \textbf{C.go},
	\textbf{B.go} y finalmente \textbf{A.go}.  \item Para compilar
	\textbf{A.go}, el compilador lee \textbf{B.go}, no \textbf{C.go}
	\end{itemize}

\section{Otras herramientas}

La distribución de Go viene con una serie de herramientas bastante útiles,
aunque todavía le faltan otras importantes, como un depurador, que está en
desarrollo. Así pues, contamos con: Godoc, Gofmt y con gdb\footnote{Aquellos que
usen \emph{gccgo} pueden invocar \textit{gdb}, pero la tabla de símbolos será
como la de C y no tendrán conocimiento del \emph{run-time}}.

	\subsection{Godoc}
	
	\emph{Godoc} es un servidor de documentación, análogo a \textbf{javadoc},
	pero más fácil para el programador que éste último.\\
	
	Como comentamos al principio del manual, la documentación oficial de Go, que
	se puede ver en su página oficial\footnote{http://golang.org} se apoya
	precisamente en esta herramienta.\\
	
	Es una herramienta muy útil para generar la documentación de nuestros
	programas y poder compartirlo con la gente de forma rápida y sencilla,
	a través de la web.\\
	
	La documentación de las librerías estándar de Go, se basa en los comentarios
	del código fuente. Para ver la documentación de una librería, podemos verla
	online a través de la web \emph{http://golang.org/pkg} o bien a través de
	la linea de comandos, ejecutando:
	
	\begin{verbatim} godoc fmt godoc fmt Printf \end{verbatim}
	
	Y así cambiando \emph{fmt} por el nombre del paquete que queramos ver,
	y \emph{Printf} por la función en cuestión a ser visualizada.
	
	\subsection{Gofmt}
	
	\emph{Gofmt} es un formateador de código. Es una herramienta muy útil para
	ver correctamente el código. Todo el código que se puede observar en la
	página web oficial, está formateado con esta herramienta.
	
	%%%%%%%%%%%%%%%%%%%%%%
	% ¿Hablamos de la herramienta gotest?    %
	%%%%%%%%%%%%%%%%%%%%%%


