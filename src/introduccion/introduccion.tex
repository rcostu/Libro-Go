% Capítulo 1: Introducción
\chapter{Introducción}

Este libro es el resultado del esfuerzo dedicado durante muchas horas a la
documentación del lenguaje de programación Go. Esta versión sustituye a todos
los efectos a la versión anterior desarrollada dentro del marco de un curso
ofrecido por uno de los autores para ACM Capítulo de Estudiantes en la Facultad de
Informática de la Universidad Politécnica de Madrid en el año 2010.\\

A través de los distintos capítulos, epígrafes y anexos se pretende iniciar al
lector en el lenguaje de programación Go, desarrollado por Google
\texttrademark, realizando una suave curva de iniciación e incluyendo una gran
cantidad de ejemplos explicativos, así como referencias a la documentación
oficial y herramientas que han sido también traducidas por los autores de este
libro.

El libro está dirigido a aquellas personas y programadores que tengan inquietud
por aprender este nuevo lenguaje de programación y que tienen una cierta base en
el mundo de la programación.

\section{¿Qué es Go?}

Go es un lenguaje de programación que fue presentado el 11 de Noviembre de 2009
y que originalmente estaba orientado a la programación de sistemas pero que ha
terminado siendo un lenguaje de programación de propósito general.\\

Según la página oficial \cite{Golang} es un lenguaje
expresivo, concurrente y que tiene recolector de basura. Además, presume de ser un
lenguaje simple, rápido, seguro, divertido y \emph{open
source}\footnote{Que el lenguaje sea \emph{Open Source} implica que cualquier
persona puede realizar modificaciones en el lenguaje o implementar nuevas
librerías que sean distribuidas como parte del mismo.}.\\

Go sigue una sintaxis bastante habitual en los lenguajes de programación más
usados, por lo que si se ha programado antes en C o Java la curva de aprendizaje
será mucho más suave.\\

Las principales características de Go son:

\begin{itemize}
	\item Es un lenguaje compilado muy, muy rápido.
	\item Usa una codificación UTF-8 para todos los ficheros fuente, es decir, permite usar
caracteres latinos, chinos, etc.  
	\item Usa tipado fuerte y memoria virtual segura.
	\item Posee punteros, pero no aritmética de los mismos.
	\item Es un lenguaje 100\% concurrente.
	\item Es Open Source, con lo que cualquier persona puede colaborar en su desarrollo
aportando ideas o implementando nuevas librerías.
	\item Por este último punto, posee una gran cantidad de librerías como, por
	ejemplo, un servidor web empotrado en el propio lenguaje.
\end{itemize}

\section{¿Quién lo desarrolla?}

Go es un proyecto promovido por cinco personas: Rob Pike, Robert Griesemer y Ken
Thompson, en primera instancia, a los que se unieron posteriormente Russ Cox
e Ian Lance Taylor. Todos los anteriormente citados, forman parte de
Google\texttrademark. Varios de ellos desarrollaron el Sistema Operativo
\emph{Plan 9} y han retomado muchas de las ideas originales para la creación
de este nuevo lenguaje de programación.

\section{¿Por qué crear un nuevo lenguaje?}

El lector podrá preguntarse cuál es la razón de crear un nuevo lenguaje de
programación, existiendo ya una gran variedad de lenguajes que permiten
desarrollar las técnicas de los grandes paradigmas de programación\cite{PVanRoy}.\\

Para empezar, el mundo informático ha avanzado enormemente en la última década,
a la par que no han aparecido nuevos lenguajes de programación que resuelvan las
nuevas casuísticas, como la programación web paralela o los problemas de
computación de altas prestaciones.\\

Actualmente, nos podemos encontrar las siguientes problemáticas que Go trata de
resolver de una manera eficiente:

\begin{itemize} 
	\item Los ordenadores son mucho más rápidos, pero no así el desarrollo de software. 
	\item Los sistemas software tienen una gran dependencia, por lo que a la
	hora de compilar es importante realizar un análisis eficiente de las
	dependencias entre los distintos ficheros, algo que no ocurre en los
	actuales ``ficheros de cabecera'' de C.
	\item Existe una tendencia creciente al uso de lenguajes de tipado dinámico,
	como Python y Javascript.
	\item La recolección de basura o la computación paralela, no están soportadas
	adecuadamente por los lenguajes de sistemas más populares.
	\item El aumento del número de núcleos en los ordenadores, ha provocado
	confusión y quebraderos de cabeza respecto a la programación concurrente y paralela. 
\end{itemize}

\section{Recursos}

Pese a que Go es un lenguaje moderno, ya existen numerosos sitios
de información sobre el lenguaje, aunque no siempre son fáciles de encontrar
debido a que el término ``Go'' es muy común en inglés. Por ello, la mayoría de
los desarrolladores de Go realizan búsquedas utilizando el término
\emph{golang}, que devuelve más y mejores resultados.

	\subsection{golang.org}
	
	El sitio web oficial del lenguaje, conviene recordar su dirección:
	\url{http://golang.org}. Toda la información que exista sobre el
	lenguaje se puede encontrar ahí, incluyendo enlaces a documentación externa
	aportada por miembros de la comunidad.\\
	
	Como curiosidad, cabe comentar que la propia página web está hecha en Go,
	utilizando el servidor web empotrado y una serie de templates HTML que trae
	``de serie''. A continuación, un repaso rápido a los sitios más importantes
	dentro de la página oficial:
	
	\begin{description} 
		\item[http://golang.org/doc] Acceso al listado de los ficheros de documentación.
		\item[http://golang.org/cmd] Acceso a la documentación sobre los comandos que pueden usarse.
		\item[http://golang.org/pkg] Acceso a la documentación de todos los paquetes existentes en Go.
		\item[http://golang.org/src] Acceso al código fuente de distintos
		ficheros de apoyo.
	\end{description}
	
	\subsection{blog.golang.org}

	El blog oficial de los desarrolladores del lenguaje. En él se van
	introduciendo entradas en las que se explican distintos aspectos del
	lenguaje de forma detallada, así como noticias que estén relacionadas con
	Go.

	\subsection{A tour of Go}

	En la dirección \url{http://go-tour.appspot.com} se encuentra una magnífica
	herramienta que permite introducir al programador en el lenguaje siguiendo
	un tutorial paso por paso. Esa es la versión original en inglés, pero se
	puede encontrar una versión traducida por los autores de este libro en
	\url{http://go-tour-es.appspot.com}

	\subsection{Lista de correo Go Nuts}
	
	La lista oficial de correo de Go se puede encontrar en la siguiente
	dirección:\\ \url{http://groups.google.com/group/golang-nuts}.
	Podrás darte de alta, ya que es una lista abierta, y recibir y enviar
	correos con todas las dudas que te surjan.

	\subsection{@go\_nuts}

	Existe una cuenta oficial del lenguaje en Twitter:
	\url[http://www.twitter.com/go\_nuts]{@go\_nuts} que suele publicar
	contenido interesante sobre Go.

	\subsection{\#go-nuts}
	
	Existe un canal de chat oficial en IRC para discutir acerca del lenguaje. Si
	quieres realizar una sesión \emph{live} sobre Go, entra en el canal
	\textbf{\#go-nuts} en el servidor \url{irc.freenode.net}.
	
	\subsection{Gestor de errores}
	
	Hay una página dedicada a gestionar todos los posibles errores del lenguaje,
	para ser así resueltos de forma eficiente. Es una página web pública y se
	puede encontraren: \url{http://code.google.com/p/go/issues/list}.
	
	\subsection{http://go-lang.cat-v.org/}
	
	Un sitio web mantenido a partir de todas las aportaciones de la gente
	a través de la lista de correo oficial. Contiene muchas librerías
	actualmente en desarrollo, ports del lenguaje a otros entornos, y sobre
	todo, archivos de coloreado de código fuente para una gran cantidad de
	programas de edición de texto.	

\section{¿Cómo instalar Go?}

Para instalar todas las librerías de Go y las herramientas propias del lenguaje,
hay que seguir unos sencillos pasos. En las próximas dos secciones se podrá ver
cómo instalarlo en sistemas UNIX y en sistemas Windows.

\subsection{Instalación en sistemas UNIX}

	Originalmente Go fue lanzado de manera exclusiva para sistemas UNIX. Aquí se
	describen los pasos necesarios para su instalación tanto en entornos Linux
	como Mac OS X y otros derivados de UNIX.

	Comenzamos con la inicialización de las variables de entorno necesarias para
	que la instalación se realice de forma correcta. Esta inicialización debería
	realizarse en ficheros de configuración dependientes del tipo de terminal
	usado, siendo para bash típicamente el fichero \emph{.bashrc}.

	\begin{enumerate} 
		\item Inicializar la variable \$GOROOT que indica el directorio raíz de Go. 
		Típicamente es el directorio \$HOME/go, aunque puede usarse cualquier otro.
		\item Inicializar las variables \$GOOS y \$GOARCH. Indican la
		combinación de Sistema Operativo y Arquitectura utilizada. Los posibles
		valores son los que se observan en la tabla \ref{Tabla_entorno}.
	
		\begin{table}[] 
			\begin{center} 
				\begin{tabular}{ccc} 
					\textbf{\$GOOS}	& \textbf{\$GOARCH} & \textbf{SO y Arquitectura}\\
					\hline 
					darwin & 386 & Mac OS X 10.5 o 10.6 -  32-bit x86\\
					darwin & amd64 & Mac OS X 10.5 o 10.6 - 64-bit x86\\ 
					freebsd & 386 & FreeBSD - 32-bit x86\\
					freebsd & amd64	& FreeBSD - 64-bit x86\\
					linux & 386 & Linux - 32-bit x86\\ 
					linux & amd64 & Linux - 64-bit x86\\
					linux & arm & Linux - 32-bit arm\\
					nacl & 386 & Native Client - 32-bit x86\\
				\end{tabular} 
			\end{center}
			\caption{Tabla de valores de \$GOOS y \$GOARCH.\label{Tabla_entorno}} 
		\end{table}
	
		\item Inicializar la variable \$GOBIN (opcional), que indica dónde serán
		instalados los binarios ejecutables. Por defecto es \$GOROOT/bin. Tras la
		instalación, es conveniente agregarla al \$PATH, para poder usar las
		herramientas desde cualquier directorio. 
	\end{enumerate}

	\nota{Hay que tener en cuenta que las variables \$GOOS y \$GOARCH
	indican el sistema contra el que se va a programar, no el sistema sobre el que
	se está programando, que estará definido por la variable \$GOHOSTOS, que
	aunque típicamente sea el mismo, no tiene por qué serlo.
	Es decir, estaremos todo el rato realizando una compilación
	cruzada.\footnote{Una compilación cruzada es aquella que se realiza cuando
	los ficheros binarios generados van a ser ejecutados en una máquina que
	posee una Arquitectura o Sistema Operativo distinto a la máquina sobre la
	que se desarrolla.}}

	Todo lo anterior se resume editando el fichero .bashrc, o cualquiera
	equivalente, añadiendo al final del fichero las siguientes líneas, con la
	configuración correcta:

\begin{lstlisting}[numbers=none]
export GOROOT=$HOME/go
export GOARCH=amd64
export GOOS=linux
export GOBIN=$GOROOT/bin
\end{lstlisting}

	Para bajarse las herramientas del repositorio, hay que tener instalado mercurial
	(tener el comando \emph{hg}). Si no se tiene instalado mercurial, se puede
	instalar con el siguiente comando:

\begin{lstlisting}[numbers=none]
$ sudo easy_install mercurial
\end{lstlisting}

	Si no se consigue instalar, conviene visitar la página oficial de descarga de
	mercurial.\footnote{http://mercurial.selenic.com/wiki/Download}\\

	Tras asegurarnos de que la variable \$GOROOT apunta a un directorio que no
	existe o que se encuentre vacío, hacemos un \emph{checkout} del repositorio
	con el comando:

\begin{lstlisting}[numbers=none]
$ hg clone -r release https://go.googlecode.com/hg/ $GOROOT
\end{lstlisting}

	Una vez que nos hemos bajado todos los ficheros necesarios, hay que instalar Go.
	Para ello hay que compilar el código fuente que nos hemos descargado
	y necesitaremos una serie de herramientas. A saber:

	\begin{itemize}
		\item GCC
		\item La biblioteca estándar de C
		\item Bison
		\item GNU make (versión $>=$ 3.81)
		\item awk
		\item ed
	\end{itemize}

	En entornos Mac OS X estas herramientas se pueden conseguir instalando el
	paquete XCode\cite{XCode}.

	El entornos Linux la instalación de estos paquetes dependerá de la
	distribución utilizada. En distribuciones basadas en Debian, se puede
	ejecutar el siguiente comando:

\begin{lstlisting}[numbers=none]
$ sudo apt-get install awk bison gcc libc6-dev ed make
\end{lstlisting}

	Para realizar la compilación e instalación de Go en base a las variables de
	entorno que definimos anteriormente, basta con ejecutar:

\begin{lstlisting}[numbers=none]
$ cd $GOROOT/src
$ ./all.bash
\end{lstlisting}

	Si la ejecución de \emph{all.bash} funciona correctamente, finalizará
	imprimiendo como últimas líneas:

\begin{lstlisting}[numbers=none]
--- cd ../test
N known bugs; 0 unexpected bugs

ALL TEST PASSED

---
Installed Go for linux/amd64 in /home/go.
Installed commands in /home/go/bin.
The compiler is 6g.
\end{lstlisting}

	donde N es un número que varía de una distribución a otra de Go, indicando bugs
	conocidos pero que no han podido ser arreglados. En la sección
	\ref{manteniendose} hablaremos sobre las distintas distribuciones de
	Go.

	Además, si estamos en Mac OS X habrá que ejecutar el comando
	\emph{./sudo.bash} para instalar los depuradores.

	\subsection{Instalación en sistemas Windows}

	La mejor forma de conseguir tener un entorno de desarrollo de Go en Windows es
	utilizar el entorno MinGW\cite{MinGW}. MinGW, de forma similar a Cygwin, simula
	un entorno tipo UNIX en sistemas operativos Windows, pero no intenta emular un
	entorno POSIX, sino que en su lugar utiliza el \emph{runtime} de C y la API de
	Windows\cite{MiekGieben}.

	Se puede por tanto instalar siguiendo los siguientes pasos, extraídos de
	\cite{MiekGieben}:

	\begin{itemize}
		\item Descargamos la última versión de MinGW de \cite{MinGW}.
		\item Extraemos los ficheros en nuestro disco C:\textbackslash.
		\item Aseguramos que los ficheros están en C:\textbackslash MingGW.
		\item Creamos un fichero dentro del directorio C:\textbackslash MinGW
		llamado \emph{setup.sh}, con el siguiente contenido:

\begin{lstlisting}[numbers=none]
export GOROOT=/c/go
export GOBIN=$GOROOT/bin
export PATH=$PATH:$GOBIN
\end{lstlisting}

		\item Ejecuta el acceso directo \emph{mintty} para lanzar una ventana de
		terminal.
		\item Cada vez que ejecutemos \emph{mintty}, podemos inicializar
		fácilmente nuestro entorno de desarrolo de Go ejecutando:

\begin{lstlisting}[numbers=none]
$ . ./setup.sh
\end{lstlisting}

		\item Descargamos la última versión estable de Go con el siguiente
		comando:

\begin{lstlisting}[numbers=none]
$ hg clone -r release https://go.googlecode.com/hg/ $GOROOT
\end{lstlisting}

		\item Al igual que en la versión UNIX, compilamos el código fuente para
		tener todo listo.

\begin{lstlisting}[numbers=none]
$ cd $GOROOT/src
$ ./all.bash
\end{lstlisting}

	Si la ejecución de \emph{all.bash} funciona correctamente, finalizará
	imprimiendo como últimas líneas:

\begin{lstlisting}[numbers=none]
--- cd ../test
N known bugs; 0 unexpected bugs

ALL TEST PASSED

---
Installed Go for linux/amd64 in /home/go.
Installed commands in /home/go/bin.
The compiler is 6g.
\end{lstlisting}
	\end{itemize}

\section{Manteniéndose al día\label{manteniendose}}

Go es un lenguaje moderno que se actualiza periódicamente. Para mantenerse al
día y conseguir que tu distribución funcione correctamente, hay que actualizar
cada vez que salga una nueva distribución, que se anuncia en la lista de correo
oficial de Go. Para actualizar a la última distribución disponible, hay que
ejecutar los siguientes comandos:

\begin{lstlisting}[numbers=none]
$ cd $GOROOT/src
$ hg pull
$ hg update release
$ ./all.bash
\end{lstlisting}

En Go existen diferentes distribuciones dependiendo del tipo de usuario al que
van dirigidas. Este libro está basado en la última versión release (r.60.3), que
es la recomendada para el público general. Las distribuciones de Go se
clasifican de la siguiente forma:

\begin{description}
	\item[Release] Es la versión más estable de Go. Se actualiza aproximadamente
	cada dos meses y es la recomendada para el público general.
	\item[Weekly] Es la versión más utilizada por los desarrolladores de
	bibliotecas del lenguaje. Se actualiza semanalmente y es propensa a errores,
	por lo que no es la más aconsejable salvo que desarrolles módulos del propio
	lenguaje.
	\item[Tip] Es la versión para los arriesgados. La que se actualiza con cada
	cambio realizado en el lenguaje y la más propensa a errores. Su uso está
	totalmente desaconsejado salvo que se necesiten los últimos cambios en el
	nucleo del lenguaje.
\end{description}

Si se quiere cambiar de una distribución a otra, se puede realizar mediante el
comando \emph{hg update distro}, donde \emph{distro} será el nombre de la
distribución que se quiera utilizar.

\newpage
