
\chapter{Modelo de Memoria de Go}

Este capítulo contiene información que afecta específicamente a la programación
concurrente en Go. Especialmente se centra en cómo afectan las operaciones a las
lecturas de datos de las variables en una goroutine de forma que se observe el
valor correcto producido por una escritura de la misma variable en otra
goroutine.

\section{Lo que primero ocurre}

En el ámbito de una única goroutine, las lecturas y escrituras deben comportarse
como si fueran ejecutadas en el orden especificado por el programa. Esto es, los
compiladores y procesadores pueden reordenar las lecturas y escrituras
ejecutadas en el ámbito de una única goroutine sólo cuando la reordenación no
cambia el comportamiento en esa goroutine definido por la especificación del
lenguaje. Debido a esta reordenación, el orden de ejecución observado por una
goroutine puede diferir del orden percibido por otra. Por ejemplo, si una
goroutine ejecuta \textit{a = 1; b=1;}, otra puede observar el valor actualizado
de \textit{b} antes que el de \textit{a}.\\

Para especificar los requisitos de lecturas y escrituras, definimos ''lo que
primero ocurre'', un orden parcial en la ejecución de operaciones de memoria en
un programa escrito en Go. Si el evento \textit{e$_1$} ocurre antes que otro
\textit{e$_2$}, entonces decimos que \textit{e$_2$} ocurre después que
\textit{e$_1$}. También, si \textit{e$_1$} no ocurre antes que \textit{e$_2$}
y no ocurre después que \textit{e$_2$}, entonces decimos que \textit{e$_1$}
y \textit{e$_2$} ocurren concurrentemente.\\

En una única goroutine, el orden de ''lo que primero ocurre'' es el orden
expresado en el programa.\\

La lectura \textit{r} de una variable \textit{v} puede observar una escritura
\textit{w} en \textit{v} si las siguientes dos afirmaciones ocurren:

\begin{enumerate} \item \textit{w} ocurre antes que \textit{r}.  \item No hay
otra escritura \textit{w} en \textit{v} que ocurra después de \textit{w} pero
antes que \textit{r}.  \end{enumerate}

Para garantizar que una lectura \textit{r} de una variable \textit{v} observa un
valor particular escrito por \textit{w} en \textit{v}, hay que asegurar que
\textit{w} es la única escritura que \textit{r} puede observar. Esto es,
\textit{r} está garantizado que lee la escritura \textit{w} si las siguientes
afirmaciones se cumplen:

\begin{enumerate} \item \textit{w} ocurre antes que \textit{r}.  \item Cualquier
otra escritura en la variable compartida \textit{v} ocurre o bien antes que
\textit{w} o bien después que \textit{r}.  \end{enumerate}

Este par de condiciones es más estricto que el primero; requiere que no haya
otras escrituras ocurriendo concurrentemente con \textit{w} o con \textit{r}.\\

En una única goroutine no existe concurrencia, así pues las dos definiciones son
equivalentes: una lectura \textit{r} observa el valor escrito por la escritura
más reciente \textit{w} en \textit{v}. Cuando varias goroutines acceden a una
variable compartida \textit{v}, deben utilizar eventos de sincronización para
establecer qué es lo que ocurre primero y así asegurar que las lecturas observan
los valores correctos de las variables compartidas.\\

La inicialización de una variable \textit{v} con el valor ''cero'' para el tipo
de \textit{v} se comporta como se ha definido en este modelo de memoria..\\

Las lecturas y escrituras de valores que ocupan más que el ancho de palabra de
la máquina, se comportan como varias operaciones del ancho de la palabra de
forma no especificada.

\section{Sincronización}

La sincronización entre las distintas goroutines es básica para que el programa
en cuestión pueda ejecutar correctamente y dar resultados satisfactorios.

	\subsection{Inicialización}
	
	La inicialización del programa se ejecuta en una única goroutine y las
	nuevas goroutines creadas durante la inicialización no comienzan su
	ejecución hasta que la inicialización termina.\\
	
	Si un paquete \textit{p} importa un paquete \textit{q}, la inicialización de
	las funciones de \textit{q} terminan antes de que empiece la inicialización
	de cualquiera de las de \textit{p}.\\
	
	La ejecución de la función \textit{main.main} comienza después de que todas
	las funciones \textit{init} hayan terminado.\\
	
	La ejecución de todas las goroutines creadas durante la ejecución de las
	funciones \textit{init} ocurre después de que todas las demás funciones
	\textit{init} hayan terminado.\\ \clearpage \subsection{Creación de
	Goroutines}
	
	La sentencia \textit{go} que comienza una nueva goroutine se ejecuta antes
	de que la ejecución de la propia goroutine comience. Veamos un ejemplo al
	respecto:
	
	\begin{verbatim} var a string
	    
		func f() { print (a) }
	    
		func hello() { a = ''Hello, World!" go f() } \end{verbatim}
	
	Una posible llamada a \textit{hello} imprimirá \textit{''Hello World!''} en
	algún momento en el futuro (quizá después de que \textit{hello} haya
	terminado).
	
	\subsection{Comunicación mediante canales}
	
	La comunicación mediante canales es el método principal de sincronización
	entre goroutines. Cada send en un canal particular se corresponde con un
	receive en ese mismo canal, que se ejecuta normalmente en otra goroutine.\\
	
	Un send en el canal se ejecuta antes de que la ejecución del receive
	correspondiente de dicho canal finalice.\\
	
	Este programa garantiza la impresión de \textit{''Hello, World!''}. La
	escritura de \textit{a} ocurre antes del send en \textit{c}, que ocurre
	antes de que el correspondiente receive en \textit{c} termine, cosa que
	ocurre siempre antes que \textit{print}.
	
	\begin{verbatim} var a string
	    
		func f() { a = ''Hello, World!" c <- 0 }
	    
		func main() { go f() <- c print (a) } \end{verbatim}
	
	\textit{Un receive de un canal sin buffer  ocurre siempre antes de que el
	send en ese canal termine}.\\
	
	El siguiente programa también garantiza que se imprime \textit{''Hello,
	World!}. La escritura en \textit{a} ocurre antes que el receive en
	\textit{c}, lo que ocurre antes de que se realice el correspondiente send
	\textit{c}, lo que ocurre antes de que se ejecute el \textit{print}.
	
	\begin{verbatim} var c = make (chan int) var a string
	    
		func f() { a = ''Hello, World!" <- c }
	    
		func main() { go f() c <- 0 print (a) } \end{verbatim}
	
	Si el canal tuviera buffer (por ejemplo, \textit{c = make (chan int, 1)})
	entonces el programa no podría garantizar la impresión de \textit{''Hello,
	World!''}. Puede imprimir la cadena vacía; No puede imprimir
	\textit{''Hello, sailor''}, aunque tampoco puede terminar de forma
	inesperada.

\section{Cerrojos - Locks}

El paquete \textit{sync} implementa dos tipos de cerrojos, \textit{sync.Mutex}
y \textit{sync.RWMutex}.\\

Para cualquier variable \textit{l} de tipo \textit{sync.Mutex}
o \textit{sync.RWMutex} y para n < m, la n-ésima llamada a \textit{l.Unlock()}
ocurre antes de que la m-ésima llamada a \textit{l.Lock()} termine.\\

El siguiente programa garantiza la impresión de \textit{''Hello, World!"}. La
primera llamada a \textit{l.Unlock()} (en \textit{f}) ocurre antes de que la
segunda llamada a \textit{l.Lock()} (en \textit{main}) termine, lo que ocurre
antes de que se ejecute \textit{print}.

\begin{verbatim} var l sync.Mutex var a string
    
	func f() { a = ''Hello, World!'' l.Unlock() }
    
	func main() { l.Lock() go f() l.Lock() print(a) } \end{verbatim}

Para cualquier llamada a \textit{l.RLock} en una variable \textit{l} de tipo
\textit{sync.RWMutex}, existe una n tal que la llamada a \textit{l.RLock}
termina después que la n-ésima llamada a \textit{l.Unlock} y su correspondiente
\textit{l.RUnlock} ocurre antes de la n+1-ésima llamada a \textit{l.Lock}.

\section{El paquete once}

El paquete \textit{once} provee de un mecanismo seguro para la inicialización en
presencia de varias goroutines. Múltiples threads pueden ejecutar
\textit{once.Do(f)} para una \textit{f} particular, pero sólo una de ellas
ejecutará la función \textit{f()} y el resto de llamadas se bloquearán hasta que
\textit{f()} haya terminado su ejecución.\\

\textit{Una única llamada a f() desde once.Do(f) termina antes de que cualquier
otra llamada a once.Do(f) termine}.\\

En el siguiente programa la llamada a \textit{twoprint} provoca que
\textit{''Hello, World!''} se imprima dos veces. La primera llamada
a \textit{twoprint} ejecuta \textit{setup} una vez.

\begin{verbatim} var a string
    
	func setup() { a = ''Hello, World!'' }
    
	func doprint() { once.Do(setup) print(a) }
    
	func twoprint() { go doprint() go doprint() } \end{verbatim}

\section{Sincronización incorrecta}

En este apartado se va a tratar los problemas más comunes que pueden darse en
una sincronización.\\

Nótese que una lectura \textit{r} puede observar el valor escrito por una
escritura \textit{w} que ocurre concurrentemente con \textit{r}. Aunque esto
ocurra, no implica que las lecturas que ocurran después de \textit{r} observen
escrituras que hayan ocurrido antes que \textit{w}.\\

En el siguiente programa puede ocurrir que \textit{g} imprima \textit{2} y luego
\textit{0}.  \clearpage \begin{verbatim} var a,b int
    
	func f() { a = 1 b = 2 }
    
	func g() { print(b) print(a) }
    
	func main() { go f() g() } \end{verbatim}

La realización de un doble intento de bloqueo para conseguir exclusión mutua, es
un intento de eliminar la sobrecarga de la sincronización. Por ejemplo, el
programa \textit{twoprint} puede ser escrito incorrectamente como se puede
observar a continuación, pero no garantiza que, en \textit{doprint}, observando
la escritura en la variable \textit{done} no implica observar la escritura en
\textit{a}. Esta versión puede (incorrectamente) imprimir una cadena vacía en
lugar de \textit{''Hello, World!''}.

\begin{verbatim} var a string var done bool
    
	func setup() { a = ''Hello, World!'' done = true }
    
	func doprint() { if !done { once.Do(setup) } print(a) }
    
	func twoprint() { go doprint() go doprint() } \end{verbatim} \clearpage Otro
	ejemplo de una mala sincronización es:

\begin{verbatim} var a string var done bool
    
	func setup() { a = ''Hello, World!'' done = true }
    
	func main() { go setup() for !done { } print(a) } \end{verbatim}
    
De la misma forma de antes no existe garantía que en \textit{main}, observando
la escritura en \textit{done} no implica observar la escritura en \textit{a},
así que este programa también puede imprimir la cadena vacía. Peor aún, no hay
garantías de que la escritura en \textit{done} sea observable desde
\textit{main}, dado que no hay eventos de sincronización entre los dos threads.
Asimismo el bucle en \textit{main} no garantiza su finalización.\\
    
Hay sutiles variantes en este tema, como este programa:

\begin{verbatim} type T struct { msg string }
    
	var g *T
    
	func setup() { t := new (T) t.msg = ''Hello, World!'' g = t }
    
	func main() { go setup() for g == nil { } print (g.msg) } \end{verbatim}

Incluso si el \textit{main} observa \textit{g != nil} y sale de su bucle, no hay
garantía de que observe el valor inicializado de \textit{g.msg}\\

Para terminar, veamos un pequeño ejemplo de una incorrecta sincronización
mediante canales. El siguiente programa, de la misma forma que los anteriores,
puede no imprimir \textit{''Hello, World!''}, imprimiendo una cadena vacía dado
que la goroutina no termina su ejecución antes de que lo haga el \textit{main}.

\begin{verbatim} var c chan int

	func hello() { fmt.Println(''Hello, World!'') c <- 0 }
    
	func main() { go hello() } \end{verbatim}

Dado que se realiza un send al canal pero no se realiza su receive
correspondiente en main, no se garantiza que hello termine de ejecutar antes de
que termine la ejecución de main, que aborta el programa.
