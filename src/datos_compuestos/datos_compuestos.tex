
\chapter{Tipos de datos compuestos\label{datos_compuestos}}

Existen una serie de tipos de datos denominados compuestos, ya se son tipos de
datos más complejos que los vistos anteriormente. Básicamente veremos Arrays,
Slices y Structs, entre otras cosas.

\section{Arrays}

Los arrays son una estructura de datos que permiten tener una serie de datos del
mismo tipo distribuidos uniformemente en un bloque de memoria.\\

Los arrays de Go son más cercanos a los de Pascal que a los de C. Más adelante,
veremos los slices que son más parecidos a los arrays de C.\\

La declaración de los arrays se realiza con la palabra reservada \textit{var}
acompañada del nombre de la variable, y el tamaño y tipo de datos que tendrá el
array. Todos los arrays deben tener un tamaño explícito.

\begin{verbatim} var ar [3]int \end{verbatim}

La anterior declaración declarará un array de nombre \textit{ar} con capacidad
para 3 números enteros. Ya que no han sido inicializados, por defecto serán 0.\\

Si en algún momento deseamos averiguar el tamaño de un array, podemos hacer uso
de la función \textit{len()}.

\begin{verbatim} len(ar) == 3 \end{verbatim}

Los arrays son valores, no punteros implícitos como ocurre en C. De todas
formas, se puede obtener la dirección de memoria donde se ha almacenado un
array, que podría servir para pasar un array de forma eficiente a una función.
Veamos un ejemplo:

%\begin{program}
\begin{verbatim} func f(a [3]int) { fmt.Println(a) }
   
   func fp(a *[3]int) { fmt.Println(a) }
   
   func main() { var ar [3] int f(ar)    // Pasa una copia de ar fp(&ar)  //
   Pasa un puntero a ar } \end{verbatim}
%\caption{Ejemplo de arrays} \end{program}

La salida del ejemplo anterior sería la siguiente:

\begin{verbatim} [0 0 0] &[0 0 0] \end{verbatim}

La función Println conoce la estructura de un array y cuando detecta uno, lo
imprime de forma óptima.\\

Los arrays también tienen su propio literal, es decir, su forma de representar
el valor real de un array. Veamos un pequeño ejemplo:

%\begin{program}
\begin{verbatim} // Array de 3 enteros [3]int { 1, 2, 3 } 
   
   // Array de 10 enteros, los 3 primeros no nulos [10]int { 1, 2, 3 }
   
   // Si no queremos contar el número de elementos // '...' lo hace por
   nosotros.  [...]int { 1, 2, 3 }
   
   // Si no se quieren inicializar todos, // se puede usar el patrón
   'clave:valor' [10]int { 2:1, 3:1, 5:1, 7:1 } \end{verbatim}
%\caption{Ejemplo de arrays literales} \end{program}

Siguiendo las normas sobre los arrays que sabemos hasta ahora, podemos conseguir
la dirección de memoria de un array literal para tener un puntero a una nueva
instancia recien creada:

%\begin{program}
\begin{verbatim} func fp(a *[3]int) { fmt.Println(a) }
   
   func main() { for i := 0; i < 3; i++ { fp(&[3]int {i, i*i, i*i*i})
   } } \end{verbatim}
%\caption{Ejemplo de punteros a arrays literales} \end{program}

El resultado de la ejecución de dicho programa, nos imprimirá por pantalla:

\begin{verbatim} &[0 0 0] &[1 1 1] &[2 4 8] \end{verbatim}

\section{Slices}

Un \textit{slice} es una \textbf{referencia} a una sección de un array. Los
slices se usan más comúnmente que los arrays.\\

Un \textit{slice} se declara como un array nada más que éste no tiene tamaño
asociado:

\begin{verbatim} var a []int \end{verbatim}

Si queremos obtener el número de elementos que posee un \textit{slice},
recurrimos a la función \textit{len()}, de igual forma que si fuera un array.\\

Un \textit{slice} puede crearse ``troceando'' un array u otro \textit{slice}.

\begin{verbatim} a = ar[7:9] \end{verbatim}

Esa instrucción nos generaría un slice a partir de un array, tomando los índices
7 y 8 del array. El número de elementos devuelto por \textit{len(a) == 2}, y los
índices válidos para acceder al slice \textit{a}, son \textit{0} y \textit{1}.\\

De igual forma, podemos inicializar un \textit{slice} asignándole un puntero
a un array:

\begin{verbatim} a = &ar     // Igual que a = ar[0:len(ar)] \end{verbatim}

Según un reciente cambio en la sintaxis referente a los slices, no es necesario
poner implícitamente los dos valores, de comienzo y fin, sino que poniendo
únicamente el valor de comienzo, nos creará un slice hasta el final del array
o slice que estemos referenciando.

\begin{verbatim} a = ar[0:]  // Igual que a = ar[0:len(ar)] b = ar[5:]  // Igual
que b = ar[5:len(ar)] \end{verbatim}

Al igual que los arrays, los slices tienen sus literales correspondientes, que
son iguales pero no tienen tamaño asociado.

\begin{verbatim} var slice = []int {1, 2, 3, 4, 5} \end{verbatim}

Lo que esta instrucción hace, es crear un array de longitud 5, y posteriormente
crea un \textit{slice} que referencia el array.

Podemos también reservar memoria para un slice (y su array correspondiente) con
la función predefinida \textit{make()}:

\begin{verbatim} var s100 = make([]int, 100)  // slice: 100 enteros
\end{verbatim}

¿Por qué usamos \textit{make()} y no \textit{new()}? La razón es que necesitamos
construir un slice, no sólo reservar la memoria necesaria. Hay que tener en
cuenta que \textit{make([]int)} devuelve \textit{[]int}, mientras que
\textit{new([]int)} devuelve \textit{*[]int)}.\\

La función \textit{make()} se utiliza para slices, maps (que veremos en un
momento) y para canales de comunicación.\\

Un \textit{slice} se refiere a un array asociado, con lo que puede haber
elementos más allá del final de un slice que estén presentes en el array. La
función \textit{cap()} (capacity) nos indica el número de elementos que el slice
puede crecer. Veamos un ejemplo:

%\begin{program}
\begin{verbatim}

   var ar = [10]int {0,1,2,3,4,5,6,7,8,9} var a = &ar[5:7]    // Referencia al
   subarray {5, 6}
   
   // len(a) == 2 y cap(a) == 5. Se puede aumentar el slice:
   
   a = a[0:4]  // Referencia al subarray {5,6,7,8}
   
   // Ahora: len(a) == 4. cap(a) == 5 .  \end{verbatim}
%\caption{Ejemplo de slices} \end{program}

¿Cómo es posible que cap(a) == 5? El aumento del slice puede ser de hasta
5 elementos. Teniendo en cuenta que si aumentamos el slice con \textit{a[0:5]},
conseguiríamos un subarray con los valores del 5 al 9.\\

Los slices pueden ser utilizados como arrays crecientes. Esto se consigue
reservando memoria para un slice con la función \textit{make()} pasándole dos
números - longitud y capacidad - y aumentándolo a medida que crezca:

%\begin{program}
\begin{verbatim}

   var sl = make([]int, 0, 100)  // Len == 0, cap == 100
   
   func appendToSlice(i int, sl []int) []int { if len(sl) == cap(sl)
   { error(...) } n := len(sl) sl = sl[0:n+1]  // Aumentamos el tamaño 1 unidad
   sl[n] = i return sl } \end{verbatim}
%\caption{Ejemplo de slices crecientes} \end{program}

La longitud de \textit{sl} siempre será el número de elementos y crecerá según
se necesite. Este estilo es mucho más ``barato'' e idiomático en Go.\\

Para terminar con los slices, hablaremos sobre lo ``baratos'' que son. Los
slices son una estructura muy ligera que permiten al programador generarlos
y aumentarlos o reducirlos según su necesidad. Además, son fáciles de pasar de
unas funciones a otras, dado que no necesitan una reserva de memoria extra.\\

Hay que recordar que un \textit{slice} ya es una referencia de por sí, y que por
lo tanto, el almacenamiento en memoria asociado puede ser modificado. Por poner
un ejemplo, las funciones de E/S utilizan slices, no contadores de elementos.
Así pues, la función Read:

%\begin{program}
\begin{verbatim}

   func Read(fd int, b []byte) int{ var buffer[100]byte for i := 0; i < 100; i++
   { // Rellenamos el buffer con un byte cada vez Read(fd, buffer[i:i+1])  //
   Aquí no hay reserva de memoria.  } }
   
   // Igualmente, podemos partir un buffer: header, data := buf[0:n],
   buf[n:len(buf)] header, data := buf[0:n], buf[n:]  // Equivalente
   \end{verbatim}
%\caption{Ejemplo de uso de slices} \end{program}

De la misma forma que un buffer puede ser troceado, se puede realizar la misma
acción con un string con una eficiencia similar.

\section{Maps}

Los \textit{maps} son otro tipo de referencias a otros tipos. Los \textit{maps}
nos permiten tener un conjunto de elementos ordenados por el par ``clave:valor",
de tal forma que podamos acceder a un valor concreto dada su clave, o hacer una
asociación rápida entre dos tipos de datos distintos. Veamos cómo se declararía
un \textit{map} con una clave de tipo \textit{string} y valores de tipo
\textit{float}:

\begin{verbatim} var m map[string] float \end{verbatim}

Este tipo de datos es análogo al tipo \textit{*map$<$string,float$>$} de C++
(Nótese el *). En un \textit{map}, la función \textit{len()} devuelve el número
de claves que posee.\\

De la misma forma que los \textit{slices} una variable de tipo \textit{map} no
se refiere a nada. Por lo tanto, hay que poner algo en su interior para que
pueda ser usado. Tenemos tres métodos posibles:

\begin{enumerate} \item Literal: Lista de pares ``clave:valor'' separados por
comas.  \begin{verbatim} m = map[string] float { "1":1, "pi:3.1415
} \end{verbatim} \item Creación con la función \textit{make()} \begin{verbatim}
m = make(map[string] float)  // recordad: make, no new.  \end{verbatim} \item
Asignación de otro map \begin{verbatim} var m1 map[string] float m1 = m  // m1
y m ahora referencian el mismo map.  \end{verbatim} \end{enumerate}

	\subsection{Indexando un map}
	
	Imaginemos que tenemos el siguiente map declarado:
	
	\begin{verbatim} m = map[string] float { "1":1, "pi":3.1415 } \end{verbatim}
	
	Podemos acceder a un elemento concreto de un map con la siguiente
	instrucción, que en caso de estar el valor en el map, nos lo devolverá,
	y sino provocará un error.
	
	\begin{verbatim} uno   := m["1"] error := m["no presente"]  //error
	\end{verbatim}
	
	Podemos de la misma forma, poner un valor a un elemento. En caso de que nos
	equivoquemos y pongamos un valor a un elemento que ya tenía un valor previo,
	el valor de dicha clave se actualiza.
	
	\begin{verbatim} m["2"] = 2 m["2"] = 3   // m[2] vale 3 \end{verbatim}
	
	\subsection{Comprobando la existencia de una clave}
	
	Evidentemente, visto el anterior punto, no es lógico andarse preocupando que
	la ejecución de un programa falle al acceder a una clave inexistente de un
	\textit{map}. Para ello, existen métodos que permiten comprobar si una clave
	está presente en un \textit{map} usando las asignaciones multivalor que ya
	conocemos, con la llamada fórmula ``coma ok". Se denomina así, porque en una
	variable se coge el valor, mientras que se crea otra llamada \textit{ok} en
	la que se comprueba la existencia de la clave.
	
	\begin{verbatim} m = map[string] float { "1":1, "pi":3.1415 }
	    
		var value float var present bool
	    
		value, present = m[x]
	    
		// o lo que es lo mismo v, ok := m[x]  // He aquí la fórmula "coma ok"
		\end{verbatim}
	
	Si la clave se encuentra presente en el \textit{map}, pone la variable
	booleana a \textit{true} y el valor de la entrada de dicha clave en la otra
	variable. Si no está, pone la variable booleana a \textit{false} y el valor
	lo inicializa al valor ``cero" de su tipo.
	
	\subsection{Borrando una entrada}
	
	Borrar una entrada de un \textit{map} se puede realizar mediante una
	asignación multivalor al map, de igual forma que la comprobación de la
	existencia de una clave.
	
	\begin{verbatim} m = map[string] float { "1":1, "pi":3.1415 }
	    
		var value float var present bool var x string = f()
	    
		m [x] = value, present \end{verbatim}
	
	Si la variable \textit{present} es \textit{true}, asigna el valor \textit{v}
	al \textit{map}. Si \textit{present} es \textit{false}, elimina la entrada
	para la clave \textit{x}. Así pues, para borrar una entrada:
	
	\begin{verbatim} m[x] = 0, false \end{verbatim}
	
	\subsection{Bucle for y range}
	
	El bucle \textit{for} tiene una sintaxis especial para iterar sobre arrays,
	slices, maps, y alguna estructura que veremos más adelante. Así pues,
	podemos iterar de la siguiente forma:
	
	\begin{verbatim} m := map[string] float { "1":1.0, "pi":3.1415 } for key,
	value := range m { fmt.Printf("clave \%s, valor \%g\n", key, value)
	} \end{verbatim}
	
	Si sólo se pone una única variable en el \textit{range}, se consigue
	únicamente la lista de claves:
	
	\begin{verbatim} m := map[string] float { "1":1.0, "pi":3.1415 } for key :=
	range m { fmt.Printf("key \%s\n", key, value) } \end{verbatim}
	
	Las variables pueden ser directamente asignadas, o declaradas como en los
	ejemplos anteriores. Para arrays y slices, lo que se consigue son los
	índices y los valores correspondientes.

\section{Structs}

Las \textit{structs} son un tipo de datos que contienen una serie de atributos
y permiten crear tipos de datos más complejos. Su sintaxis es muy común, y se
pueden declarar de dos formas:

\begin{verbatim} var p struct { x, y float }
    
	// O de forma más usual type Punto struct { x, y float } var p Point
	\end{verbatim}

Como casi todos los tipos en Go, las \textit{structs} son valores, y por lo
tanto, para lograr crear una referencia o puntero a un valor de tipo
\textit{struct}, usamos el operador \textit{new(StructType)}. En el caso de
punteros a estructuras en Go, no existe la notación $->$, sino que Go ya provee
la indirección al programador.

\begin{verbatim} type Point struct { x, y float } var p Point p.x = 7 p.y = 23.4
var pp *Point = new(Point) *pp = p pp.x = Pi    // equivalente a (*pp).x
\end{verbatim}

Ya que los \textit{structs} son valores, se puede construir un \textit{struct}
a ``cero'' simplemente declarándolo. También se puede realizar reservando su
memoria con \textit{new()}.

\begin{verbatim} var p Point        // Valor a "cero" pp := new(Point);  //
Reserva de memoria idiomática \end{verbatim}

Al igual que todos los tipos en Go, los \textit{structs} también tienen sus
literales correspondientes:

\begin{verbatim} p = Point { 7.2, 8.4 } p = Point { y:8.4, x:7.2 } pp := &Point
{ 23.4, -1 } // Forma correcta idiomáticamente \end{verbatim}

De la misma forma que ocurría con los arrays, tomando la dirección de un
\textit{struct} literal, da la dirección de un nuevo valor creado. Este último
ejemplo es un constructor para *Point.

	\subsection{Exportación de tipos}
	
	Los campos (y métodos, que veremos posteriormente) de un \textit{struct}
	deben empezar con una letra Mayúscula para que sea visible fuera del
	paquete. Así pues, podemos tener las siguientes construcciones de
	\textit{struct}:
	
	\begin{itemize} \item Tipo y atributos privados: \begin{verbatim} type point
	struct { x, y float } \end{verbatim} \item Tipo y atributos exportados:
	\begin{verbatim} type Point struct { X, Y float } \end{verbatim} \item Tipo
	exportado con una mezcla de atributos: \begin{verbatim} type point struct
	{ X, Y float      // exportados nombre string   // privado } \end{verbatim}
	\item Podríamos incluso tener un tipo privado con campos exportados.
	\end{itemize}
	
	\subsection{Atributos anónimos}
	
	Dentro de un \textit{struct} se pueden declarar atributos, como otro
	\textit{struct}, sin darle ningún nombre al campo. Este tipo de atributos
	son llamados atributos anónimos y actúan de igual forma que si la
	\textit{struct} interna estuviera insertada o ``embebida" en la externa.\\
	
	Este mecanismo permite una manera de derivar algunas o todas tus
	implementaciones desde cualquier otro tipo o tipos. Veamos un ejemplo:
	
	\begin{verbatim} type A struct { ax, ay int }
	    
		type B struct { A bx, by float } \end{verbatim}
	
	\textit{B} actúa como si tuviera cuatro campos: \textit{ax}, \textit{ay},
	\textit{bx} y \textit{by}. Es casi como si \textit{B} hubiera sido declarada
	con cuatro atributos, dos de tipo int, y otros dos de tipo float. De todas
	formas, los literales de tipo \textit{B}, deben ser correctamente
	declarados:
	
	\begin{verbatim} b := B { A { 1, 2 }, 3.0, 4.0 } fmt.Println( b.ax, b.ay,
	b.bx, b.by ) \end{verbatim}
	
	Pero todo el tema de los atributos anónimos es mucho más útil que una simple
	interpolación de los atributos: \textit{B} también tiene el atributo
	\textit{A}. Un atributo anónimo se parece a un atributo cuyo nombre es su
	tipo:
	
	\begin{verbatim} b := B { A { 1, 2 }, 3.0, 4.0 } fmt.Println( b.A
	) \end{verbatim}
	
	El código anterior imprimiría {1 2}. Si \textit{A} es importado de otro
	paquete, el atributo aún así debe seguir llamándose A:
	
	\begin{verbatim} import "pkg" type C struct { pkg.A } ...  c := C { pkg.A{
	1, 2 } } fmt.Println(c.A)  // No c.pkg.A \end{verbatim}
	
	Para terminar, cualquier tipo nombrado, o puntero al mismo, puede ser
	utilizado como un atributo anónimo y puede aparecer en cualquier punto del
	\textit{struct}.
	
	\begin{verbatim} type C struct { x float int string }
	    
		c := C { 3.5, 7, "Hola" } fmt.Println(c.x, c.int, c.string)
		\end{verbatim}
	
	\subsection{Conflictos}
	
	Pueden darse una serie de conflictos dada la posible derivación de unos
	\textit{structs} en otros. De hecho, la mayoría de conflictos vendrán dados
	por tener dos atributos distintos pero con el mismo nombre. Para ello, se
	aplican las siguientes reglas en caso de que esto suceda:
	
	\begin{enumerate} \item Un atributo ``externo'' oculta un atributo
	``interno". Esto da la posibilidad de sobreescribir atributos y métodos.
	\item Si el mismo nombre aparece dos veces en el mismo nivel, se provoca un
	error si el nombre es usado por el programa. Si no es usado, no hay ningún
	problema. No existen reglas para resolver dicha ambigüedad, con lo que debe
	ser arreglado.  \end{enumerate}
	
	Veamos algún ejemplo:
	
	\begin{verbatim} type A struct { a int } type B struct { a, b int }
	    
		type C struct { A; B } var c C \end{verbatim}
	
	Usando \textit{c.a} es un error. ¿Lo sería \textit{c.A.a} o \textit{c.B.a}?
	Evidentemente no, ya que son perfectamente accesibles.
	
	\begin{verbatim} type D struct { B; b float } var d D \end{verbatim}
	
	El uso de \textit{d.b} es correcto: estamos accediendo al float, no
	a \textit{d.B.b}. La \textit{b} interna se puede conseguir accediendo
	a través de \textit{d.B.b}.
